\chapter{Zero-Knowledge Proofs}
\label{chap:zkproofs}

We now give a brief introduction to \glspl{zkproof}.



\section{The Need for Zero-Knowledge Proofs}

The purpose of \glspl{zkproof} is to \emph{prove}
something is true without revealing \emph{any additional information}.

We note that all \glspl{signature} are a form of \gls{zkproof}.
As discussed in Chapter~\ref{chap:signatures},
If Alice signs a message $m$ resulting in signature $\sigma$
and sends $\parens{m,\sigma}$ to Bob,
he is able to prove to Charlie that Alice did send him the message $m$.
In this case, Charlie sees the message along with the signature
and knows that Alice signed the message and sent the pair to Bob,
but he does not learn any additional information;
in particular, he does not gain the ability to forge
signatures of Alice.
\Glspl{zkproof} attempt to generalize this to proofs about
other knowledge.



\section{Group for Examples}
\label{sec:zkproofs_group}

Throughout this chapter we will be using a particular
\gls{finite cyclic group} for the examples.
We have the following:

\begin{align}
    p &= rq + 1 \nonumber\\
    q &= 2^{32} + 15 \nonumber\\
    r &= 12.
\end{align}

\noindent
It can be shown that $p$ and $q$ are prime numbers.
Additionally, it can also be shown that

\begin{equation}
    h = 7
\end{equation}

\noindent
is a generator for $\F_{p}^{*}$.
Thus, we set

\begin{equation}
    g = h^{r} \mod p.
\end{equation}

\noindent
From this, we know that $\abs{\angles{g}} = q$.

When we need two generators, we set

\begin{align}
    h_{1} &= 7 \nonumber\\
    h_{2} &= 6
\end{align}

\noindent
and

\begin{align}
    g_{1} &= h_{1}^{r} \mod p \nonumber\\
    g_{2} &= h_{2}^{r} \mod p.
\end{align}

\noindent
It can also be shown that $6$ is also a generator of $\F_{p}^{*}$;
see Appendix~\ref{app:math_finite_fields_primitive} for more information.



\section{Proving knowledge of Discrete Logarithms}

We now work through \glspl{zkproof} involving
\glspl{discrete log};
this section uses examples found in~\cite{GeneralDiscreteLogProofs}.

Throughout we suppose $G = \angles{g}$ is a \gls{finite cyclic group}
of order $q$.
Furthermore, we suppose that we have a \gls{hash function}
$H:\braces{0,1}^{*}\to\braces{0,1}^{b}$.

\subsection{Knowledge of Discrete Logarithm}

\subsubsection{Definition}

We let $x\chooseRandom{}\Z_{q}^{*}$ and set $y = g^{x}$.
To prove knowledge of $x$, 

\begin{enumerate}
\item We let $v\chooseRandom{}\Z_{q}$ and set $t = g^{v}$;
    $t$ is the \emph{commitment}.
\item We have the \emph{challenge}

\begin{equation}
    c = H(g,y,t).
\end{equation}

\item We have the \emph{response}

\begin{equation}
    r = v - cx \mod q.
\end{equation}
\end{enumerate}

The challenge-response pair $\parens{c,r}$ can be verified by anyone.
The proof involves constructing the commitment

\begin{equation}
    t' = g^{r}y^{c}
\end{equation}

\noindent
and verifying

\begin{equation}
    c = H(g,y,t').
\end{equation}

\noindent
As noted in~\cite{GeneralDiscreteLogProofs},
this is essentially a Schnorr signature of the message $\parens{g,y}$.
Schnorr signatures are discussed in Chapter~\ref{sec:signatures_schnorr}.

\subsubsection{Verification}

If $\parens{c,r}$ is as defined, then we see

\begin{align}
    t' &= g^{r}y^{c}
            \nonumber\\
        &= g^{v-cx}\cdot\parens{g^{x}}^{c}
            \nonumber\\
        &= g^{v}\cdot g^{-cx}\cdot g^{xc}
            \nonumber\\
        &= g^{v}
            \nonumber\\
        &= t.
\end{align}

\noindent
Thus, we have

\begin{align}
    H(g,y,t') &= H(g,y,t) \nonumber\\
        &= c.
\end{align}

\subsubsection{Example}

\begin{example}[Knowledge of Discrete Logarithm]
\exampleCodeReference{examples/zk\_proofs/dl\_proof\_1.py}

We use the \gls{cyclic group} described in Chapter~\ref{sec:zkproofs_group};
see Listing~\ref{list:zkproof_dl_proof_1}.

\paragraph{Secret Information}
In this case, we let

\begin{align}
    g &= 13841287201 \nonumber\\
    y &= 34063925739.
\end{align}

\noindent
We have

\begin{equation}
    y = g^{x} \mod p
\end{equation}

\noindent
with

\begin{equation}
    x = 710124910.
\end{equation}

\paragraph{Setup Challenge-Response}
To make our challenge-response, we set

\begin{align}
    v &= 979360098 \nonumber\\
    t &= g^{v} \mod p \nonumber\\
        &= 26735074229.
\end{align}

\noindent
This allows us to compute

\begin{align}
    c &= \text{\MDFive{}}\parens{g,y,t} \mod q \nonumber\\
        &= 2451489814 \nonumber\\
    r &= v - cx \mod q \nonumber\\
        &= 3624355949.
\end{align}

\noindent
This implies that $\parens{2451489814,3624355949}$
is the challenge-response for knowledge of the \gls{discrete log} for
$\parens{13841287201,34063925739}$.

\paragraph{Validate Challenge-Response}
To validate the \gls{zkproof}, we see

\begin{align}
    t' &= g^{r}y^{c} \mod p \nonumber\\
        &= 26735074229 \nonumber\\
    c' &= \text{\MDFive{}}\parens{g,y,t'} \mod q \nonumber\\
        &= 2451489814.
\end{align}

\noindent
The proof is valid because $c=c'$.

\lstset{
    showstringspaces=false,
    frame=single
}
\lstinputlisting[
    label=list:zkproof_dl_proof_1,
    float,
    language=Python,
    caption=Zero-knowledge proof of knowledge of discrete logarithm]{code/zk_proofs/dl_proof_1.py}

\end{example}

\subsubsection{Intuition}

We take some time to try to explain what is happening intuitively.

We are building a challenge-response test where the challenge
is a pseudorandom number (from a \gls{hash function}).
For an arbitrary $\parens{c',r'}$ pair, we have

\begin{align}
    t' &= g^{r'}y^{c'} \nonumber\\
        &= g^{r'+c'x}.
\end{align}

\noindent
If $x$ is unknown, then this will appear to be a random element of $G$.
Given a \gls{hash function} (with a sufficiently large output),
it should be computational infeasible to ensure

\begin{equation}
    c' = H(g,y,t')
\end{equation}

\noindent
unless $x$ is known.


\subsection{Knowledge of Two Discrete Logarithms}
\label{sec:zkproofs_2_dlog}

\subsubsection{Definition}

In this case, we assume $g_{1}$ and $g_{2}$ are independent generators of $G$.
For $x\chooseRandom{}\Z_{q}^{*}$, we set $y_{1} = g_{1}^{x}$ and
$y_{2} = g_{2}^{x}$.
We wish to prove that $x$ is known.

We proceed in a manner similar as before:

\begin{enumerate}
\item We let $v\chooseRandom{}\Z_{q}$ and set $t_{1} = g_{1}^{v}$
    and $t_{2} = g_{2}^{v}$.
\item We have the \emph{challenge}

\begin{equation}
    c = H(g_{1},y_{1},g_{2},y_{2},t_{1},t_{2}).
\end{equation}

\item We have the \emph{response}

\begin{equation}
    r = v - cx \mod q.
\end{equation}
\end{enumerate}

The challenge-response pair $\parens{c,r}$ can be verified by anyone.
The proof involves constructing the commitment

\begin{align}
    t_{1}' &= g_{1}^{r}y_{1}^{c} \nonumber\\
    t_{2}' &= g_{2}^{r}y_{2}^{c}
\end{align}

\noindent
and confirming

\begin{equation}
    c = H(g_{1},y_{1},g_{2},y_{2},t_{1}',t_{2}').
\end{equation}

\subsubsection{Verification}

If $\parens{c,r}$ is as defined, then we see

\begin{align}
    t_{i}' &= g_{i}^{r}y_{i}^{c}
            \nonumber\\
        &= g_{i}^{v-cx}\cdot\parens{g_{i}^{x}}^{c}
            \nonumber\\
        &= g_{i}^{v}\cdot g_{i}^{-cx}\cdot g_{i}^{xc}
            \nonumber\\
        &= g_{i}^{v}
            \nonumber\\
        &= t_{i}.
\end{align}

\noindent
Thus, we have

\begin{align}
    H(g_{1},y_{1}, g_{2},y_{2},t_{1}',t_{2}')
        &= H(g_{1},y_{1}, g_{2},y_{2},t_{1},t_{2}) \nonumber\\
        &= c.
\end{align}

\subsubsection{Example}

\begin{example}[Knowledge of Two Discrete Logarithms]
\exampleCodeReference{examples/zk\_proofs/dl\_proof\_2.py}

We use the \gls{cyclic group} described in Chapter~\ref{sec:zkproofs_group};
see Listing~\ref{list:zkproof_dl_proof_2}.

\paragraph{Secret Information}
In this case, we let

\begin{align}
    g_{1} &= 13841287201
        \nonumber\\
    g_{2} &= 2176782336
        \nonumber\\
    y_{1} &= 34063925739
        \nonumber\\
    y_{2} &= 42077908773.
\end{align}

\noindent
We have

\begin{align}
    y_{1} &= g_{1}^{x}
        \nonumber\\
    y_{2} &= g_{2}^{x}
\end{align}

\noindent
with

\begin{equation}
    x = 710124910.
\end{equation}

\paragraph{Setup Challenge-Response}
To make our challenge-response, we set

\begin{align}
    v &= 979360098 \nonumber\\
    t_{1} &= g_{1}^{v} \mod p \nonumber\\
        &= 26735074229 \nonumber\\
    t_{2} &= g_{2}^{v} \mod p \nonumber\\
        &= 14645668233.
\end{align}

\noindent
This allows us to compute

\begin{align}
    c &= \text{\MDFive{}}\parens{g_{1},y_{1},g_{2},y_{2},t_{1},t_{2}} \mod q
            \nonumber\\
        &= 820804763 \nonumber\\
    r &= v - cx \mod q \nonumber\\
        &= 1837574845.
\end{align}

\noindent
This implies that $\parens{820804763,1837574845}$
is the challenge-response for knowledge of the two \glspl{discrete log}
$\parens{13841287201,34063925739}$ and $\parens{2176782336,42077908773}$.

\paragraph{Validate Challenge-Response}
To validate the \gls{zkproof}, we see

\begin{align}
    t_{1}' &= g_{1}^{r}y_{1}^{c} \mod p \nonumber\\
        &= 26735074229 \nonumber\\
    t_{2}' &= g_{2}^{r}y_{2}^{c} \mod p \nonumber\\
        &= 14645668233 \nonumber\\
    c' &= \text{\MDFive{}}\parens{g_{1},y_{1},g_{2},y_{2},t_{1}',t_{2}'} \mod q
            \nonumber\\
        &= 820804763.
\end{align}

\noindent
The proof is valid because $c=c'$.

\lstset{
    showstringspaces=false,
    frame=single
}
\lstinputlisting[
    label=list:zkproof_dl_proof_2,
    float,
    language=Python,
    caption=Zero-knowledge proof of knowledge of two discrete logarithms]{code/zk_proofs/dl_proof_2.py}

\end{example}


\subsection{Knowledge of Multiple Discrete Logarithms}

The situation for two values can be generalized to multiple values.

\subsection{Knowledge of Linear Combination of Two Discrete Logarithms}

\subsubsection{Definition}

We now suppose that $g_{1}$ and $g_{2}$ are distinct generators of $G$
and  $x_{1},x_{2}\chooseRandom{}\Z_{q}^{*}$.
We want to prove that

\begin{align}
    y_{1} &= g_{1}^{x_{1}} \nonumber\\
    y_{2} &= g_{2}^{x_{2}} \nonumber\\
    a_{1}x_{1} + a_{2}x_{2} &= b \mod q
\end{align}

\noindent
for specified $\parens{a_{1},a_{2}}\in\Z_{q}^{2}$.
That is, we know the \glspl{discrete log} and a linear combination of them.
This is more involved than the previous proofs.

\begin{enumerate}
\item Let 

\begin{equation}
    \parens{v_{1},v_{2}}\chooseRandom{}
        \braces{\parens{u_{1},u_{2}}\in\Z_{q}^{2}
            \mid a_{1}u_{1} + a_{2}u_{2} = 0 \mod q}.
\end{equation}

We have the commitments

\begin{align}
    t_{1} &= g_{1}^{v_{1}} \nonumber\\
    t_{2} &= g_{2}^{v_{2}}.
\end{align}

\item We have the challenge

\begin{equation}
    c = H(g_{1},y_{1},g_{2},y_{2},a_{1},a_{2},b,t_{1},t_{2}).
\end{equation}

\item We have the responses

\begin{align}
    r_{1} &= v_{1} - cx_{1} \mod q \nonumber\\
    r_{2} &= v_{2} - cx_{2} \mod q.
\end{align}
\end{enumerate}

\noindent
The proof is $\parens{c,r_{1},r_{2}}$.

The verify the proof, we set

\begin{align}
    t_{1}' &= g_{1}^{r_{1}}y_{1}^{c} \nonumber\\
    t_{2}' &= g_{2}^{r_{2}}y_{2}^{c}.
\end{align}

\noindent
The proof is valid if

\begin{equation}
    c = H(g_{1},y_{1},g_{2},y_{2},a_{1},a_{2},b,t_{1}',t_{2}')
\end{equation}

\noindent
and

\begin{equation}
    a_{1}r_{1} + a_{2}r_{2} = -cb \mod q.
\end{equation}

\subsubsection{Verification}

We now show that if $\parens{c,r_{1},r_{2}}$ is as defined
then the proof is valid.

First, we see

\begin{align}
    t_{i}' &= g_{i}^{r_{i}}y_{i}^{c}
            \nonumber\\
        &= g_{i}^{v_{i}-cx_{i}}\cdot\parens{g_{i}^{x_{i}}}^{c}
            \nonumber\\
        &= g_{i}^{v_{i}}
            \nonumber\\
        &= t_{i}.
\end{align}

\noindent
Thus, the commitments are valid, so we have that

\begin{align}
    H(g_{1},y_{1},g_{2},y_{2},a_{1},a_{2},b,t_{1}',t_{2}').
        &= H(g_{1},y_{1},g_{2},y_{2},a_{1},a_{2},b,t_{1},t_{2})
            \nonumber\\
        &= c.
\end{align}

\noindent
We next see that

\begin{align}
    a_{1}r_{1} + a_{2}r_{2}
        &= a_{1}\parens{v_{1}-cx_{1}} + a_{2}\parens{v_{2}-cx_{2}} \mod q
            \nonumber\\
        &= \parens{a_{1}v_{1} + a_{2}v_{2}} - c\parens{a_{1}x_{1} + a_{2}x_{2}}
                \mod q
            \nonumber\\
        &= -c\parens{a_{1}x_{1} + a_{2}x_{2}} \mod q
            \nonumber\\
        &= -cb \mod q.
\end{align}

\noindent
Between the first and second lines, we rearrange terms.
The first term from the second line is zero from the assumptions
on $\parens{v_{1},v_{2}}$.
The fourth line follows from the third by assumption as well.

\subsubsection{Example}

\begin{example}[Knowledge of Linear Combination of Two Discrete Logarithms]
\exampleCodeReference{examples/zk\_proofs/dl\_proof\_linear.py}

We use the \gls{cyclic group} described in Chapter~\ref{sec:zkproofs_group};
see Listing~\ref{list:zkproof_dl_proof_linear}.
We note that some of the output is missing due to its length.

\paragraph{Secret Information}
In this case, we let

\begin{align}
    g_{1} &= 13841287201
        \nonumber\\
    g_{2} &= 2176782336
        \nonumber\\
    y_{1} &= 34063925739
        \nonumber\\
    y_{2} &= 6808289188
\end{align}

\noindent
with

\begin{align}
    y_{1} &= g_{1}^{x_{1}}
        \nonumber\\
    y_{2} &= g_{2}^{x_{2}}
\end{align}

\noindent
for

\begin{align}
    x_{1} &= 710124910
        \nonumber\\
    x_{2} &= 285758093.
\end{align}

\noindent
We have

\begin{equation}
    a_{1}x_{1} + a_{2}x_{2} = b
\end{equation}

\noindent
with

\begin{align}
    a_{1} &= 521
        \nonumber\\
    a_{2} &= 523
        \nonumber\\
    b &= 4030483429.
\end{align}

\paragraph{Setup Challenge-Response}
To make our challenge-response, we set

\begin{align}
    v_{1} &= 979360098 \nonumber\\
    v_{2} &= 2333891425 \nonumber\\
    t_{1} &= g_{1}^{v_{1}} \mod p \nonumber\\
        &= 26735074229 \nonumber\\
    t_{2} &= g_{2}^{v_{2}} \mod p \nonumber\\
        &= 47139203830.
\end{align}

\noindent
This allows us to compute

\begin{align}
    c &= \text{\MDFive{}}\parens{g_{1},y_{1},g_{2},y_{2},a_{1},a_{2},b,
            t_{1},t_{2}} \mod q \nonumber\\
        &= 1236818951 \nonumber\\
    r_{1} &= v_{1} - cx_{1} \mod q \nonumber\\
        &= 1952959998 \nonumber\\
    r_{2} &= v_{2} - cx_{2} \mod q \nonumber\\
        &= 3581019185.
\end{align}

\noindent
This implies that $\parens{1236818951,1952959998,3581019185}$
is the challenge-response for knowledge of the linear combination
of \glspl{discrete log}.

\paragraph{Validate Challenge-Response}
To validate the \gls{zkproof}, we see

\begin{align}
    t_{1}' &= g_{1}^{r_{1}}y_{1}^{c} \mod p \nonumber\\
        &= 26735074229 \nonumber\\
    t_{2}' &= g_{2}^{r_{2}}y_{2}^{c} \mod p \nonumber\\
        &= 47139203830 \nonumber\\
    c' &= \text{\MDFive{}}\parens{g_{1},y_{1},g_{2},y_{2},a_{1},a_{2},b,
            t_{1}',t_{2}'} \mod q \nonumber\\
        &= 1236818951.
\end{align}

\noindent
We see that $c=c'$ and that

\begin{align}
    a_{1}r_{1} + a_{2}r_{2} \mod q &= 4147159721 \nonumber\\
    -cb \mod q &= 4147159721.
\end{align}

\noindent
Thus, we have

\begin{equation}
    a_{1}r_{1} + a_{2}r_{2} = -cb \mod q.
\end{equation}

\noindent
The proof is valid because $c=c'$ and the correct linear combination.

\lstset{
    showstringspaces=false,
    frame=single
}
\lstinputlisting[
    label=list:zkproof_dl_proof_linear,
    float,
    language=Python,
    caption=Zero knowledge proof of knowledge of linear combination of two discrete logarithms]{code/zk_proofs/dl_proof_linear.py}

\end{example}

\subsection{Knowledge of Linear Combination of Multiple Discrete Logarithms}

We can easily generalize the results from the previous section
to a linear combination of $n$ \gls{discrete log} values.



\section{Concluding Discussion}

We walked through some basic examples of \glspl{zkproof}
by focusing on proving knowledge of \glspl{discrete log}.
There is much more which could be said on this topic
but we will not at this time.
See~\cite{GeneralDiscreteLogProofs} for more information
and a more thorough discussion.

We note some extensions of \glspl{zkproof} include
Non-Interactive Zero-Knowledge Proofs
(NIZKs)~\cite{rackoff1991nizk,santis1987nizk},
Zero-Knowledge Succinct Non-interactive ARguments of Knowledge
(zk-SNARKs)~\cite{cryptoeprint:2011:443},
and Zero-Knowledge Scalable Transparent ARguments of Knowledge
(zk-STARKs)~\cite{cryptoeprint:2018:046,cryptoeprint:2018:1098}.
The interest in \glspl{zkproof} may be related
to interest in \gls{ethereum}~\cite{EthereumYellowpaper}
and related cryptocurrencies.

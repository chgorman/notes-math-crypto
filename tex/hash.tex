\chapter{Cryptographic Hash Functions}
\label{chap:hash}

This chapter is devoted to introducing \glsfirstplural{hash function}.
\Glspl{hash function} are used all over cryptography.

\emph{Note:} In Chapters~\ref{chap:hash} and \ref{chap:hash_applications},
we will frequently use the \MDFive{} and \ShaOne{} \glspl{hash function}
in the examples due to their small output size.
As explained below, \MDFive{} and \ShaOne{}
\emph{should never be used in practice}.



\section{The Need for Cryptographic Hash Functions}

In cryptography, we want to be able to work with arbitrary data;
the challenge, of course, is that it may be difficult to design
algorithms which can handle arbitrary-length data.
It would be better if, instead of working with the raw data,
we would work with a ``fingerprint'' or ``hash'' of the data;
in this way, algorithms would be able to work with identifiers
of a fixed size.
For this to be useful in a cryptographic setting,
we would want hashes to be effectively unique.
Furthermore, additional intuitive properties are desired:
each hash should be easy to compute and appear random;
the only way to determine a hash is to explicitly compute it; and
it should be hard to find two different files with the same hash.

Thus, we are interested in \glsfirstplural{hash function};
these are hash functions with additional properties to defend against
cryptographic adversaries.



\section{Desired Properties of Cryptographic Hash Functions}
\label{sec:hash_properties}

In the cryptographic setting, the \glspl{hash function} we are concerned about
are generally called \emph{\glsfirstplural{hash function}}.
We let $H:\braces{0,1}^{*}\to\braces{0,1}^{b}$ be a
\glsfirst{hash function} with an output of $b$ bits.
We want $H$ to satisfy a number of properties:

\begin{description}
\item [Easy to Compute]
    Given an input $x$, we want to be able to easily compute $H(x)$.

    \emph{Discussion:} Most \glspl{hash function} have
        some inherent message length limitations,
        but in practice these lengths are extremely long
        and so are treated as infinite.
\item [Random Output]
    Given an input $x$, we want $H(x)$ to be a ``random'' element
    of $\braces{0,1}^{b}$.

    \emph{Discussion:} One of the important assumptions
        involving \glspl{hash function} is that it should not be possible
        to obtain \emph{any} information about the output
        without explicitly computing $H(x)$.
        In particular, even if $x'$ is somehow related to $x$,
        $H(x')$ should be completely unrelated to $H(x)$
        provided $x\ne x'$.
\item [Preimage Resistance]
    Given an output $y$, it should be impractical to find $x$ such that

\begin{equation}
    H(x) = y.
\end{equation}

    \emph{Discussion:} We expect that every possible output has some input
        (in fact, many inputs) which maps to it,
        but the challenge is that it is not clear how
        to find such inputs efficiently.
        We expect that the best choice is just to try all possible inputs
        (or rather, around $2^{b}$ possible inputs) to find $x$.

\item [Second-Preimage Resistance]
    Given $x_{1}$, it should be impractical to find a distinct $x_{2}$ such that

\begin{equation}
    H(x_{1}) = H(x_{2}).
\end{equation}

    \emph{Discussion:} Given $x_{1}$, we realize it should be possible
        to find $x_{2}$ which results in the same hash output,
        but the emphasis here is that there should be no
        efficient method for finding such pairs.
        We expect the best choice would be to try around $2^{b}$ inputs
        to find $x_{2}$.

\item [Collision Resistance]
    It should be impractical to find any $x_{1}\ne x_{2}$ such that

\begin{equation}
    H(x_{1}) = H(x_{2}).
\end{equation}

    \emph{Discussion:} We expect there to be no efficient method
        to find \emph{any} pairs $x_{1}$ and $x_{2}$ that hash to the
        same output.
        We expect the best choice would be to try around $2^{b/2}$ inputs
        to find $x_{1}$ and $x_{2}$.
\end{description}

Although here we have restricted ourselves to \glspl{hash function}
from bit strings to bit strings,
it is possible to expand the definition from any \gls{set} to
any other \gls{set}.
In particular, we look at hashing to \glspl{elliptic curve} in
Chapter~\ref{chap:pairing}.
Some standard \glspl{hash function} will be discussed below
in Chapter~\ref{sec:hash_examples}.

\emph{Note well:} It is important to note that hashing data
\emph{does not}, by itself, have anything to do with encrypting
said data.
Additionally, the relationship between preimage resistance,
second-preimage resistance, and collision resistance is intricate;
see~\cite{cryptoeprint:2004:035} for an extended discussion about
the relationships between them.



\section{Random Oracle: The Ideal Cryptographic Hash Function}
\label{sec:random_oracle}

A \emph{\gls{random oracle}} is an idealized form of a
\glsfirst{hash function}.
Put another way, when cryptographers write proofs they may
assume that all parties have access to a \gls{random oracle};
in practice, the \gls{random oracle} is instantiated with a specific
\gls{hash function}.

Suppose we have a \gls{random oracle} $H:\braces{0,1}^{*}\to\braces{0,1}^{b}$.
A \gls{random oracle} $H(\cdot)$ has the property that,
for any $x\in\braces{0,1}^{*}$,
the output $H(x)$ is uniformly distributed on $\braces{0,1}^{b}$.



\section{Additional Properties}
\label{sec:additional_hash_properties}

There are additional properties that we want in a \glsfirst{hash function}:

\begin{description}
\item [Correlation Resistance]
    Even if $x$ and $y$ are correlated, $H(x)$ and $H(y)$
        should be unrelated.

    \emph{Discussion:} This provides greater detail related to
        the \textbf{Random Output} property.
        It is common that correlated inputs are hashed.
        In fact, in some instances $x$ and $y$ may vary by only \emph{one bit},
        yet we want the outputs $H(x)$ and $H(y)$ to be completely
        unrelated.
 \item [Length-extension Resistance]
    If $H(x) = H(y)$ with $x\ne y$, then for any bit string $z$
    of positive length, $H(x||z)$ and $H(y||z)$ are independent.

    \emph{Discussion:} Again, this provides greater detail related to
        the \textbf{Random Output} property.
    \emph{There are \glspl{hash function} which do not satisfy this property};
        see Chapter~\ref{ssec:hash_challenges_length_extension}.
\end{description}



\section{Examples}
\label{sec:hash_examples}

We will now spend some time looking at the outputs of various
\glspl{hash function}.
\exampleCodeReference{examples/hash\_functions/}
One way to gain some understanding about \glspl{hash function}
is to repeatedly hash data.
For each \gls{hash function}, we will look at three different examples:

\begin{enumerate}
\item ``'' (the empty byte slice);
\item ``The quick brown fox jumps over the lazy dog''
    (converted into bytes)
\item ``The quick brown fox jumps over the lazy dog.''
    (converted into bytes; note the period).
\end{enumerate}

\noindent
Although the last 2 values are very similar,
we would expect there to be drastic changes in the hash output;
this happens in all the examples below.
This drastic change is known as the \emph{avalanche effect}:
any change in input should result in a vastly different output,
as we expect the outputs to be completely unrelated.

\paragraph{\MDFive{}} Designed by Ronald Rivest
and published in 1992~\cite{rfc1321} with an output of 128 bits (16 bytes).
At this point, due to the small size and significant attacks against it,
\MDFive{} should no longer be used in any situation
where a \gls{hash function} is necessary.
Example hashes can be found in Listing~\ref{list:md5}.

\paragraph{\ShaOne{}} Designed by NIST and published
in 1995~\cite{FIPS-180-1-1995},
\ShaOne{} has an output of 160 bits (20 bytes).
Although more secure than \MDFive{},
there are also significant attacks against \ShaOne{}
and it should not be used.
Example hashes can be found in Listing~\ref{list:sha1}.

\paragraph{\ShaTwo{}} Due to the attacks against \ShaOne{},
NIST released \ShaTwo{} in 2001~\cite{FIPS-180-4-2015}.
The standard had multiple algorithms with different output sizes;
one commonly used algorithm is \ShaTwo{}-256 (also called SHA-256),
which has a 256 bit (32 byte) output.
Although still subject to certain attacks (described below),
it is still considered secure.
\ShaTwo{}-256 example hashes can be found in Listing~\ref{list:sha2}.

\paragraph{\ShaThree{}}
The continued cryptanalysis of \MDFive{}, \ShaOne{}, and \ShaTwo{}
caused NIST to have a competition (similar to the competition which
produced the block cipher AES, the Advanced Encryption Standard).
In the end, \Keccak{}~\cite{KeccakSponge2011} became the basis
for \ShaThree{}~\cite{FIPS-202}.
Like \ShaTwo{}, there are multiple versions with different output lengths;
a common one is \ShaThree{}-256 with 256 bits (32 bytes) of output.
\ShaThree{}-256 example hashes can be found in Listing~\ref{list:sha3}.

\lstset{
    showstringspaces=false,
    frame=single
}
\lstinputlisting[
    label=list:md5,
    float,
    language=Python,
    caption=Output from the MD5 hash function]{code/hash_functions/hash_md5.py}

\lstset{
    showstringspaces=false,
    frame=single
}
\lstinputlisting[
    label=list:sha1,
    float,
    language=Python,
    caption=Output from the SHA-1 hash function]{code/hash_functions/hash_sha1.py}

\lstset{
    showstringspaces=false,
    frame=single
}
\lstinputlisting[
    label=list:sha2,
    float,
    language=Python,
    caption=Output from the SHA-2-256 hash function]{code/hash_functions/hash_sha2.py}

\lstset{
    showstringspaces=false,
    frame=single
}
\lstinputlisting[
    label=list:sha3,
    float,
    language=Python,
    caption=Output from the SHA-3-256 hash function]{code/hash_functions/hash_sha3.py}




\section{Domain Separation}

It may happen that, due to resource constraints, only \emph{one}
\gls{hash function} may be implemented.
Sometimes it is desired to instantiate different \glspl{hash function}
for different uses.
We can do this with \emph{domain separation}:
adding a value to specify the specific domain.
This will ensure that the queries to the underlying \gls{hash function}
are unique regardless of input.

In particular, if $H$ is a \gls{hash function}, we can make two different
\glspl{hash function} by specifying

\begin{align}
    H_{0}(x) &\mathDef{} H(\texttt{0}||x) \nonumber\\
    H_{1}(x) &\mathDef{} H(\texttt{1}||x),
\end{align}

\noindent
where $\texttt{0}$ and $\texttt{1}$ are bits.
When viewed as \glspl{random oracle}, this is not a problem.
Care must be required when this is used in practice, though;
this is because we can run into issues when working with
\glspl{hash function} in the real world.
This is discussed more in Chapter~\ref{sec:hash_challenges}.



\section{Known Challenges with Certain Hash Functions}
\label{sec:hash_challenges}

Here we briefly discuss some of the known difficulties which may arise when
working with \glspl{hash function} in practice.

\subsection{Length Extension Attacks}
\label{ssec:hash_challenges_length_extension}

The \MDFive{}, \ShaOne{}, and \ShaTwo{} \glspl{hash function} are based
on the \MD{}
construction~\cite{merkle1979secrecy,damgaard1989design};
this is discussed in Appendix~\ref{app:crypto_md_construction}.
In this case, given the length of $x$ and $H(x)$ (but not $x$ explicitly),
there exists $y$ such that for any $z$
it is easy to compute

\begin{equation}
    H(x||y||z).
\end{equation}

\noindent
Here, $y$ is related to the internal padding of the underlying
\gls{hash function}.
Because this padding is publicly known,
it is trivial to determine it based solely on the length of the input.
This implies that one must be careful in order to use
\glspl{hash function} in \glspl{mac};
see Chapter~\ref{sec:hash_hmac} for more information.

Not all \glspl{hash function} are susceptible to this attack.
\Keccak{} and \ShaThree{} were designed to resist all known attacks;
their resistance is based on the fact that they are built
using a sponge construction
(discussed in Appendix~\ref{app:crypto_sponge_construction})
and not all their internal state is output.
Similarly, \ShaTwo{}-512/256 is not affected because its internal state
is truncated before output
(512 bit internal state truncated to 256 bits for output).
In these instances, a portion of the internal state is not output,
so total internal state recovery would require guessing at least 256 bits
thereby making this attack impractical.

\subsection{Other Attacks}

We note that the collision resistance of \MDFive{} and \ShaOne{}
are \emph{significantly below} their original security level
(respectively 64 bits and 80 bits).
It is \emph{practical} to find \MDFive{} and \ShaOne{} hash
collisions~\cite{cryptoeprint:2004:199,cryptoeprint:2005:067,
cryptoeprint:2004:356,cryptoeprint:2020:014,rfc6151,rfc6194};
preimage resistance has been attacked~\cite{MD5FastPreimage2009}.

See Listing~\ref{list:md5_collision} for an \MDFive{} hash collision
and Figure~\ref{fig:shattered} for a \ShaOne{} hash collision.

\lstset{
    showstringspaces=false,
    frame=single
}
\lstinputlisting[
    label=list:md5_collision,
    float,
    language=Python,
    caption={Example MD5 Collision from Marc Stevens~\cite{md5Collision}}]{code/hash_functions/md5_collision.py}

\begin{figure}
\centering
\begin{subfigure}{\textwidth}
    \centering
    \includegraphics[width=0.75\textwidth]{figures/hash_functions/shattered-1.pdf}
    \caption{\texttt{shatter-1.pdf}}
    \label{fig:shattered-1}
    \end{subfigure}
    \vspace{0.5cm}

    \begin{subfigure}{\textwidth}
    \centering
    \includegraphics[width=0.75\textwidth]{figures/hash_functions/shattered-2.pdf}
    \caption{\texttt{shatter-2.pdf}}
    \label{fig:shattered-2}
    \end{subfigure}

\caption[Example Files Producing \ShaOne{} Collision]{Here
    are two different files which produce the same \ShaOne{} hash value:
    \texttt{38762cf7f55934b34d179ae6a4c80cadccbb7f0a}.
    Files from \url{https://shattered.io/}.}
\label{fig:shattered}
\end{figure}



\section{Properties of a Secure Cryptographic Hash Function}

Ideally, a \gls{hash function} will be as close to a
\gls{random oracle} as possible.
So, a good \gls{hash function} will have all of the properties
from Chapters~\ref{sec:hash_properties} and
\ref{sec:additional_hash_properties}.

This is not always possible in practice.
As discussed previously, \MDFive{} and \ShaOne{} do not satisfy
all of the standard hash properties (particularly collision resistance);
this makes them insecure.
Thus, \MDFive{} and \ShaOne{} should only by used
when \emph{required} for legacy systems.

Due to length extension attacks, \ShaTwo{} cannot satisfy
all the properties of a \gls{random oracle}.
With that said, the best attacks against its collision resistance
are no where close to working on the entire \gls{hash function}
(just weaker versions).
In this way, \ShaTwo{} is secure provided that care is taken
to ensure that it is not possible to exploit length extension attacks.
By design, \ShaThree{} is not susceptible to length extension attacks.
This makes it closer to a \gls{random oracle}.

In conclusion, it would be acceptable to use either \ShaTwo{} or \ShaThree{}
in situations where a \glsfirst{hash function} is required.
If using \ShaTwo{}, ensure that it is not possible
to exploit length extension attacks.

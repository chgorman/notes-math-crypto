\section{Number Theory}

We review some basic facts of \gls{number theory} related to divisibility,
prime numbers, and modular arithmetic.
Additional material may be found in Appendix~\ref{app:math_nt}.
A more systematic treatment
may be found in~\cite{ComputationalIntroNTA}.

\subsection{Divisibility and Prime Numbers}

Given $a,b\in\Z$, we say $a$ divides $b$, written as $a\mid b$, when
$b = ca$ for $c\in\Z$;
in this case, we call $a$ a \emph{divisor} of $b$.
We write $a\nmid b$ when $a$ does not divide $b$ and say $a$ is not
a divisor of $b$.
Because $a = 1\cdot a$, we see that $1\mid a$ and $a\mid a$;
thus, $1$ and $a$ are always divisors of $a$.

If $p\in\N\setminus\braces{0,1}$ and we \emph{only} have
$1\mid p$ and $p\mid p$,
then we say that $p$ is a \emph{prime number}.
We frequently use $p$ and $q$ to denote prime numbers.
If $n\in\N\setminus\braces{0,1}$ and $n$ is not prime,
then we say that $n$ is \emph{composite}.

By convention, mathematicians do not consider $0$ or $1$
to be prime or composite numbers.

\begin{example}[Prime and Composite Numbers]
We know that $15 = 3\cdot 5$, so $15$ is a composite number.
We know that we only have $3 = 1\cdot 3$ and $5 = 1\cdot 5$,
so $3$ and $5$ are prime numbers.
\end{example}

\subsection{Greatest Common Divisor}

Given $a,b\in\Z$ (with $a\ne0$ or $b\ne0$), $d\ge0$ is the
\emph{greatest common divisor} of $a$ and $b$, written as
$\gcd(a,b) = d$, if $d$ is a divisor of $a$ and $b$ and that
if $r\ge0$ is also a divisor of $a$ and $b$ then $r\le d$.
When $\gcd(a,b) = 1$, we say that $a$ and $b$ are \emph{coprime}:
they have no common divisors.
Stated another way, coprime numbers have no common prime factors.

There are efficient algorithms to compute $\gcd(a,b)$;
see Appendix~\ref{app:math_nt} for more details.
One inefficient method involves computing the prime factorization
of $a$ and $b$ and then collecting the common factors.

\begin{example}[GCD Examples]
\label{example:bnt_gcd}
\exampleCodeReference{examples/math\_review/gcd.py}

We have $3 = 1\cdot3$ and $5 = 1\cdot5$;
this implies

\begin{equation}
    \gcd(3,5) = 1.
\end{equation}

\noindent
We know $15 = 3\cdot 5$;
this implies that

\begin{align}
    \gcd(15,5) &= 5 \nonumber\\
    \gcd(15,3) &= 3.
\end{align}

\noindent
We know $1872 = 2^{4}\cdot3^{2}\cdot13$ and
$1080 = 2^{3}\cdot3^{3}\cdot5$;
this implies that

\begin{align}
    \gcd(1872, 1080) &= 2^{3}\cdot3^{2} \nonumber\\
        &= 72.
\end{align}

\noindent
This is worked out in
Example~\ref{example:app_math_euclidean_alg}.
\end{example}

If $\gcd(a,b) = d$, then we can always find $x,y\in\Z$ such that

\begin{equation}
    ax + by = d.
\end{equation}

\noindent
There are efficient algorithms to compute this;
see Appendix~\ref{app:math_nt} for more details.

\begin{example}
\label{example:bnt_extended_gcd}
\exampleCodeReference{examples/math\_review/gcd.py}

We continue the previous example for $\gcd(1872, 1080) = 72$.
We see

\begin{equation}
    1872 \cdot (-4) + 1080 \cdot 7 = 72.
\end{equation}

\noindent
This is worked out in
Example~\ref{example:app_math_extended_euclidean_alg}.
\end{example}

\subsection{Congruent Numbers}

Let $n>1$ be an integer.
Given $a,b\in\Z$, we say that $a$ is congruent to $b$ modulo $n$
if $n\mid a-b$;
that is, we have $a-b = kn$ for some $k\in\Z$.
We also write this as $a \equiv b \mod n$.

\begin{example}[Examples of Congruent Numbers Mod 2]
We let $n=2$.
In this case, we see that $1 \equiv 11\mod 2$ because

\begin{equation}
    1 - 11 = \parens{-5}\cdot2.
\end{equation}

\noindent
Similarly, $2 \equiv 10 \mod 2$:

\begin{equation}
    2 - 10 = \parens{-4}\cdot2.
\end{equation}

\noindent
Continuing on, we see that all even numbers are congruent modulo $2$
and all odd numbers are congruent modulo $2$.

We show this now.
Let $a = 2k$ and $b = 2\ell$.
Then

\begin{equation}
    a - b = 2\parens{k-\ell}.
\end{equation}

\noindent
Thus, $a\equiv b \mod 2$.
This gives us the result for even numbers.
If $c = 2m+1$ and $d = 2n + 1$, then we see

\begin{equation}
    c - d = 2\parens{m-n}.
\end{equation}

\noindent
Thus, $c\equiv d \mod 2$.
This gives us the result for odd numbers.

We just showed that all even numbers are congruent to $0$ modulo $2$
and all odd numbers are congruent to $1$ modulo $2$.
\end{example}



\subsection{Modular Arithmetic}

Let $n>1$ be an integer.
Given $a\in\Z$, we can always compute $q, r\in\Z$ such that

\begin{equation}
    a = qn + r,
\end{equation}

\noindent
where $q$ the \emph{quotient}
and $0\le r < n$ the \emph{remainder};
this is called \emph{Euclidean Division}
and is discussed in Appendix~\ref{app:math_nt}.
From our previous discussion, we see that

\begin{equation}
    a\equiv r\mod n.
\end{equation}

\noindent
Using this fact, we will show that addition and multiplication
are always well-defined in modular arithmetic.
From there, exponentiation and other operations follow.
The only operation which requires care is division,
which we will look at separately.

\paragraph{Addition} We let $a_{1}, a_{2}\in\Z$ with

\begin{align}
    a_{1} &= q_{1}n + r_{1} \nonumber\\
    a_{2} &= q_{2}n + r_{2},
\end{align}

\noindent
where $q_{i}\in\Z$ and $0\le r_{i} < n$.
Then we see

\begin{align}
    a_{1} + a_{2} &= \parens{q_{1} + q_{2}}n + \parens{r_{1}+r_{2}}
        \nonumber\\
    &= r_{1} + r_{2} \mod n.
\end{align}

\noindent
This shows that addition behaves as we would expect.

\paragraph{Multiplication} We let $a_{1}$ and $a_{2}$ be as before.
Then

\begin{align}
    a_{1}a_{2} &= \parens{q_{1}q_{2}n + r_{1}q_{2} + r_{2}q_{1}}n + r_{1}r_{2}
        \nonumber\\
    &= r_{1}r_{2} \mod n.
\end{align}

\noindent
This shows that multiplication behaves as we would expect.

\paragraph{Division} We need to be careful in order to define division.
Within the rational or real numbers, when we say that $x$
is the multiplicative inverse of $a$,
we mean that

\begin{equation}
    ax = 1.
\end{equation}

We look for something similar in modular arithmetic.
Thus, if $a\in\Z\setminus\braces{0}$, then we look for an $x\ne0$ such that

\begin{equation}
    ax \equiv 1 \mod n.
\end{equation}

\noindent
This implies that there is $k\in\Z$ such that

\begin{equation}
    ax - 1 = -nk,
\end{equation}

\noindent
or rather

\begin{equation}
    ax + nk = 1.
\end{equation}

The above equality implies that

\begin{equation}
    \gcd(a,n) = 1.
\end{equation}

\noindent
Thus, if $\gcd(a,n) = 1$, then $a$ has a multiplicative inverse
modulo $n$.

\begin{example}[Modular Arithmetic Example]
\exampleCodeReference{examples/math\_review/modular\_arithmetic.py}

We let $n = 256$.
We have

\begin{align}
    27 + 241 &\equiv 12 \mod 256 \nonumber\\
    27 \cdot 241 &\equiv 107 \mod 256.
\end{align}

\noindent
Because $\gcd(27,256) = 1$ and $\gcd(241,256) = 1$,
$27$ and $241$ have multiplicative inverses.
We find that

\begin{align}
    27 \cdot 19 &\equiv 1 \mod 256 \nonumber\\
    241 \cdot 17 &\equiv 1 \mod 256.
\end{align}
\end{example}

\chapter{Conclusion}
\label{chap:conclusion}

The material we covered here should give an individual the necessary
background to begin to understand the mathematics behind cryptography.
Naturally, much more could be learned.
We now discuss general paths of study and additional resources.



\section{Suggestions for Further Study}

For those interested in further study, here are some suggestions
for learning cryptography at a more rigorous level.
The books are the same;
the difference is order.

\subsection{General STEM and Non-Mathematicians}

The path described here is based on the author's thoughts
and what he thinks would be helpful.

\begin{itemize}
    \item \textbf{Understanding Cryptography} by Christof Paar
    and Jan Pelzl~\cite{UnderstandingCrypto}.
    This book gives a broad introduction to cryptography.
    It includes many exercises, all of which should be completed
    or at least attempted.
    The presentation of these notes was influenced by this textbook.
\item \textbf{An Introduction to Mathematical Cryptography}
    by Jeffrey Hoffstein, Jill Pipher,
    and Joseph H.~Silverman~\cite{IntroMathCrypto}.
    This book should help by providing the opportunity to learn
    mathematics at a deeper level than what may be learned from
    \emph{Understanding Cryptography}.
    With it being more mathematical,
    it will also help individuals develop the necessary mathematical
    maturity for rigorous cryptography.
    Additional \emph{useful} mathematical material may be found
    in~\cite{ComputationalIntroNTA}.
\item \textbf{Introduction to Modern Cryptography}
    by Jonathan Katz and Yehuda Lindell~\cite{IntroModernCrypto}.
    This textbook is the standard reference for teaching \emph{rigorous}
    cryptography focused on definitions, theorems, and proofs.
    This is a challenging work and assumes a level of mathematical maturity.
\end{itemize}

\subsection{Mathematicians}

The path described here is based on the author's personal experience.
Here, we assume that an individual has the equivalent of a bachelor's
degree in mathematics and is comfortable with theorems,
definitions, and proofs;
no particular knowledge of \gls{number theory} is assumed.
The books described are the same as above but the author recommends
a different order.

\begin{itemize}
\item \textbf{An Introduction to Mathematical Cryptography}
    by Jeffrey Hoffstein, Jill Pipher,
    and Joseph H.~Silverman~\cite{IntroMathCrypto}.
    This book will help increase mathematical ability as it pertains
    to cryptography.
    It also introduces some useful probability theory.
    This will help mathematicians understand where mathematics
    fits into cryptography.
    If additional mathematical material is desired,
    see~\cite{ComputationalIntroNTA}.
\item \textbf{Understanding Cryptography} by Christof Paar
    and Jan Pelzl~\cite{UnderstandingCrypto}.
    This book gives a broad introduction to cryptography.
    Although the parts devoted to mathematics are not difficult,
    the book does an excellent job of giving an overview of the major
    areas of cryptography.
    In this way, a broad understanding of cryptography will be gained.
\item \textbf{Introduction to Modern Cryptography}
    by Jonathan Katz and Yehuda Lindell~\cite{IntroModernCrypto}.
    This textbook is the standard reference for teaching \emph{rigorous}
    cryptography focused on proofs.
    With a background in mathematics, the level of mathematical maturity
    should not be difficult;
    the challenge may be learning a new field.
\end{itemize}



\section{Online Resources}

Here are non-exhaustive lists of online resources.
The author has used some, but not all, of these resources.

\subsection{Online Cryptography Resources}
\label{ssec:conclusion_online_crypto}

\begin{itemize}
\item The Cryptopals website: \url{https://cryptopals.com/}.
    This website helps you learn cryptography by solving problems.
\item IACR website: \url{https://iacr.org/}.
    The International Association for Cryptologic Research
    is a professional society devoted to cryptography research.
    IACR also has a preprint server that has many interesting
    cryptography papers: \url{https://eprint.iacr.org/}.
\item Online cryptography notes:
    there are a number of online cryptography notes.
    These are all around the level of
    \emph{Introduction to Modern Cryptography}~\cite{IntroModernCrypto}.

    \begin{itemize}
    \item A Graduate Course in Applied Cryptography
        by Dan Boneh and Victor Shoup~\cite{BonehShoupGraduateApplied}.
        This is a work in progress that is mostly complete
        but does not have references.
    \item An Intensive Introduction to Cryptography
        by Boaz Barak~\cite{IntensiveCrypto}.
        This is a work in progress.
    \item A Course in Cryptography
        by Rafael Pass and Abhi Shelat~\cite{CourseCrypto}.
        Pass's A Course in Discrete Structures~\cite{CourseDiscreteStructures}
        may also be helpful.
        Both of these are aimed at the advanced undergraduate level.
    \item Introduction to Modern Cryptography
        by Mihir Bellare and Phillip Rogaway~\cite{BellareRogawayIMC}.
    \item Lecture Notes on Cryptography
        by Shafi Goldwasser and Mihir Bellare~\cite{GoldwasserBellareLNC}.
        Some of the material here is borrowed from
        Bellare and Rogaway~\cite{BellareRogawayIMC}.
    \end{itemize}
\item Dan Boneh has two online cryptography
    classes~\cite{BonehCourseraCryptoI,BonehCourseraCryptoII}.
\item Online references for zero-knowledge proofs and related topics:
    \begin{itemize}
    \item An online course on zero-knowledge proofs~\cite{OnlineMOOCZKProofs}
        by Dan Boneh, Shafi Goldwasser, Dawn Song, Justin Thaler,
        and Yupeng Zhang.
    \item An online book on SNARGs (Succinct Non-interactive ARGuments)
        \cite{SnargsBook} by Alessandro Chiesa and Eylon Yogev.
    \item An online book Proofs, Arguments,
        and Zero-Knowledge~\cite{ThalerProofsZK} by Justin Thaler.
    \end{itemize}
\end{itemize}

\subsection{Online Mathematics Resources}
\label{ssec:conclusion_online_math}

\begin{itemize}
\item AMS Open Math Notes~\cite{AMSOpenMathNotes}:
    The American Mathematical Society has a collection of notes
    for standard mathematics courses at both the undergraduate
    and graduate level.
    Although learning mathematics in the classroom is usually ideal,
    these notes may help individuals learn additional
    mathematics.
\item AIM Open Textbook Initiative~\cite{AIMTextbooks}:
    The American Institute of Mathematics has a list of open textbooks
    covering many subjects including abstract algebra, \gls{number theory},
    and introduction to proofs.
\end{itemize}



\section{Other Good Books}

There are many other good books on cryptography and mathematics.
Below are some books the author has purchased while trying
to learn cryptography:

\begin{itemize}
\item \textbf{A Computational Introduction to Number Theory and Algebra}
    by Victor Shoup~\cite{ComputationalIntroNTA}.
    This is a useful book that gives an introduction to both
    \gls{number theory} and abstract algebra at the undergraduate level.
    It covers a lot of useful material and may be found online.
    \emph{This book would be helpful when trying to build
    mathematical maturity.}
\item \textbf{Cryptography Engineering}
    by Niels Ferguson, Bruce Schneier, and Tadayoshi Koh\-no~\cite{CryptoEng}.
    In some ways, this is an updated version of \emph{Applied Cryptography}
    by Bruce Schneier~\cite{AppliedCrypto},
    that was written in 1996;
    although useful at the time, it is out-of-date and does not cover
    significant advances.
    Cryptography Engineering focuses on the \emph{engineering}
    aspects of cryptography rather than the mathematics.
    This is useful when implementing cryptographic algorithms in order
    to know the common pitfalls and how to avoid them.
    This is considered the second edition to
    \emph{Practical Cryptography}~\cite{PracticalCryptography}.
\item \textbf{Foundations of Cryptography} (Volumes I and II)
    by Oded Goldreich~\cite{FoundationsCrypto1,FoundationsCrypto2}.
    These books are standard textbooks on theoretical cryptography
    and rigorous proofs.
    They are more advanced than~\cite{IntroModernCrypto}.
\item \textbf{Handbook of Applied Cryptography}
    by Alfred Menezes, Paul van Oorschot,
    and Scott Vanstone~\cite{HandbookAppliedCrypto}.
    This has been a standard cryptography reference.
    Being written in 1997, it is starting to be out of date
    like \emph{Applied Cryptography}~\cite{AppliedCrypto}.
    \Glspl{elliptic curve} are only briefly mentioned.
\item \textbf{Handbook of Elliptic and Hyperelliptic Curve Cryptography}
    by Henri Cohen, Gerhard Frey, Roberto Avanzi, Christophe Doche,
    Tanja Lange, Kim Nguyen, and Frederik Vercauteren~\cite{HandbookECC}.
    This is a standard reference for \gls{ecc}.
    This text generally assumes advanced mathematical knowledge;
    \emph{do not} try to learn \gls{ecc} here.
\item \textbf{Security Engineering} by Ross Anderson~\cite{SecurityEng}.
    Although not focused on cryptography,
    this book helps place cryptography in the overall context of security.
    He talks about ways cryptographic implementations may leak information;
    this results in the notion of \emph{side-channel attacks}.
    When implementing cryptographic algorithms,
    it is \emph{critical} to not accidentally leak information
    through side-channels.
    Operations as simple as comparing bytes may leak timing information
    and result in leaking the private key.
\end{itemize}

\noindent
Of course, not all of these books need to be purchased at once.
Furthermore, the author has not yet thoroughly worked through
all of these books.
It may be good to hold off on purchasing the older books
as well as the advanced books until they are needed.



\section{Useful Areas of Mathematics and Books}

Any serious interest
in understanding advanced cryptography requires mathematics
\emph{at least} at the undergraduate level.
Here are some useful areas of mathematics and potential books for study.
In general, the goal would be to understand algebra and \gls{number theory};
some knowledge of probability theory is helpful as well.

\begin{itemize}
\item Abstract Algebra: Here we gave a gentle introduction
    to \glspl{group}, \glspl{ring}, and \glspl{field}.
    A more rigorous treatment of these algebraic objects would
    be of great value for anyone who wants to have a better understanding
    of the underlying mathematics of cryptography.
    One standard abstract algebra book is
    \emph{Abstract Algebra} by Dummit and Foote~\cite{DummitFooteAlgebra}.
    Other books may give an easier first pass on the subject, though;
    see Chapter~\ref{ssec:conclusion_online_math} for online resources,
    and~\cite{ComputationalIntroNTA} may also be useful.

\item Number Theory: as we saw, \gls{number theory} is extremely useful
    in \gls{publiccrypto}.
    The study of \glspl{elliptic curve} also generally falls under
    \gls{number theory} as well.
    After working through
    \emph{A Computational Introduction to Number Theory and Algebra}
    by Victor Shoup~\cite{ComputationalIntroNTA},
    it would be helpful to spend some time on \glspl{elliptic curve}.
    One undergraduate book is \emph{Rational Points on Elliptic Curves} by
    Silverman and Tate~\cite{RationalPointsEC}.
\item Probability Theory: although not necessarily a part of mathematics,
    additional knowledge of probability is always useful.
    One standard undergraduate textbook is
    \emph{Introduction to Probability Models} by Ross~\cite{IntroProbModels}.
\end{itemize}

More advanced areas include

\begin{itemize}
\item Elliptic Curves:
    Advanced knowledge of \gls{ecc} will
    require a deep understanding of \glspl{elliptic curve}.
    The standard graduate level textbook on \glspl{elliptic curve} is
    \emph{The Arithmetic of Elliptic Curves} by Silverman~\cite{AEC}.
    Another book that focuses on elliptic curves in cryptography
    is \emph{Elliptic Curves: Number Theory and Cryptography}
    by Lawrence Washington~\cite{WashingtonEllipticCurves}.
    Additional study of commutative algebra would probably be helpful.
\item Advanced Number Theory:
    Spending additional time learning \gls{number theory} would complement
    further study on \glspl{elliptic curve}.
    One book to start may be
    \emph{A Course in Number Theory and Cryptography}
    by Neal Koblitz~\cite{KoblitzCourseNTCrypto}.
\end{itemize}

\noindent
The books included here are meant to be useful guides when attempting
to learn advanced mathematics.
Online notes may likely be found that cover the material suggested here;
see potential mathematical notes and textbooks in
Chapter~\ref{ssec:conclusion_online_math}.
When attempting to learn mathematics through self-study,
the greatest challenge will likely be gaining mathematical maturity;
that is, the ability to write rigorous proofs, formalize intuitive logic,
and recognize invalid logic.

Good luck.

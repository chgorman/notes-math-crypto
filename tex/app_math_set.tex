\section{Set Theory}
\label{app:math_set_theory}

We now spend a bit more time discussing set theory and set operations.
We will need the following result to prove two \glspl{set} are equal:

\begin{thm}[Set Equality]
\label{thm:set_equality}
For \glspl{set} $A$ and $B$, we have $A=B$ if and only if
$A\subseteq B$ and $B\subseteq A$.
\end{thm}

\begin{proof}
Let us suppose that $A=B$, then for $x\in A$ we have $x\in B$
so $A\subseteq B$.
Similarly, for $x\in B$ we have $x\in A$ so $B\subseteq A$.

If $A\ne B$, then they do not contain the same elements.
Thus, there is either some $x\in A$ such that $x \notin B$
(implying that $A\not\subseteq B$)
or some $x\in B$ such that $x\notin A$
(implying that $B\not\subseteq A$).
In either case, we either have $A\not\subseteq B$ or $B\not\subseteq A$.
\end{proof}

\subsection{Additional Set Operations}

From Eq.~\eqref{eq:math_set_theory_basic_set_ops}, we recall

\begin{align}
    A \cup B &\mathDef{} \braces{x \mid x\in A \text{ or } x\in B}
        \nonumber\\
    A \cap B &\mathDef{} \braces{x \mid x\in A \text{ and } x\in B}
        \nonumber\\
    A \setminus B &\mathDef{} \braces{a\in A \mid a\notin B}.
\end{align}

\noindent
These are not the only set operations, though.

First, we say that $X$ is a \emph{universal set} if all of the sets
we will work with will be subsets of $X$.
When working with a universal set $X$ and a set $A\subseteq X$,
we have the \emph{set complement} or \emph{complement}:

\begin{equation}
    A^{c} \mathDef{} \braces{x\in X \mid x\notin A}.
\end{equation}

\noindent
This may also be written in terms of the set minus operation:

\begin{equation}
    X\setminus A = \braces{x\in X \mid x\notin A}.
\end{equation}

\noindent
The complement notation $A^{c}$ may be preferred to the set minus notation
$X\setminus A$ when there are many set operations in the same expression.

For $A\subseteq X$, we always have

\begin{align}
    A \cup A^{c} &= X \nonumber\\
    A \cap A^{c} &= \emptyset.
\end{align}

\begin{example}[Set Complements]
We let our universal set be the set of natural numbers $\N$.
In this case, we define the sets

\begin{align}
    A &= \braces{0,1,2,3,4} \nonumber\\
    B &= \braces{1,2,3,\cdots}.
\end{align}

\noindent
In this case, we see

\begin{align}
    A^{c} &= \braces{5,6,7,\cdots} \nonumber\\
    B^{c} &= \braces{0}.
\end{align}
\end{example}

We can also take unions and intersections in a more general way.
Specifically, let $\mathcal{A}$ be a nonempty indexing set;
that is, $\mathcal{A}$ is a set which will be used to index other sets.
Given a collection of sets $\braces{A_{\alpha}}_{\alpha\in\mathcal{A}}$,
we define

\begin{align}
    \bigcup_{\alpha\in\mathcal{A}} A_{\alpha} &\mathDef{} 
        \braces{x\mid x\in \mathcal{A}_{\alpha} \text{ for some }
        \alpha\in\mathcal{A}} \nonumber\\
    \bigcap_{\alpha\in\mathcal{A}} A_{\alpha} &\mathDef{} 
        \braces{x\mid x\in \mathcal{A}_{\alpha} \text{ for all }
        \alpha\in\mathcal{A}}.
\end{align}

\noindent
These are generalizations of the standard union and intersection operations.

\begin{example}[Set Unions and Intersections]
We now look at some set unions and intersections.
First, we will let $\N$ be our indexing set.
We see

\begin{equation}
    \bigcup_{k\in\N} \braces{k} = \N.
\end{equation}

\noindent
In this case, we are just taking the union of the set which contains
each natural number;
naturally, this is then just the natural numbers $\N$.

We also see

\begin{equation}
    \bigcap_{k\in\N} \braces{k} = \emptyset.
\end{equation}

\noindent
This example shows that, instead of taking the union,
we take the intersection of all the sets which contain each natural number,
the result is the empty set.

A more involved example is this:

\begin{equation}
    \bigcap_{k\in\N} \braces{k,k+1,k+2,\cdots} = \emptyset.
\end{equation}

\noindent
This shows that if you take the of natural numbers starting at
$k$ for each $k\in\N$ and then intersect them all,
the resulting set contains no natural numbers.
This follows from the fact that there is no largest natural number.

Intersections need not always be empty, as this next example shows:

\begin{equation}
    \bigcap_{k\in\N} \braces{-k,-k+1,-k+2,\cdots,k-2,k-1,k} = \braces{0}.
\end{equation}
\end{example}


\subsection{Relationships Between Set Operations}

We now discuss a very useful result: De Morgan's Laws.

\begin{thm}[De Morgan's Laws]
Given a nonempty index set $\mathcal{A}$, we have the following:

\begin{align}
    \brackets{\bigcup_{\alpha\in\mathcal{A}} A_{\alpha}}^{c}
        &= \bigcap_{\alpha\in\mathcal{A}} A_{\alpha}^{c} \nonumber\\
    \brackets{\bigcap_{\alpha\in\mathcal{A}} A_{\alpha}}^{c}
        &= \bigcup_{\alpha\in\mathcal{A}} A_{\alpha}^{c}.
\end{align}
\end{thm}

\noindent
This result is useful because it helps us rearrange set operations
into forms which are more amenable.

In the case where we just have one intersection and union,
De Morgan's Laws reduce to the following:

\begin{align}
    \brackets{A\cup B}^{c} &= A^{c} \cap B^{c} \nonumber\\
    \brackets{A\cap B}^{c} &= A^{c} \cup B^{c}.
\end{align}

\begin{proof}
We will use Theorem~\ref{thm:set_equality} in our proof.

Let

\begin{equation}
    x\in \brackets{\bigcup_{\alpha\in\mathcal{A}} A_{\alpha}}^{c}.
\end{equation}

\noindent
This implies that 

\begin{equation}
    x\notin \bigcup_{\alpha\in\mathcal{A}} A_{\alpha},
\end{equation}

\noindent
from which it follows that $x\notin A_{\alpha}$ for all $\alpha\in\mathcal{A}$.
Therefore we have

\begin{equation}
    x\in \bigcap_{\alpha\in\mathcal{A}} A_{\alpha}^{c}.
\end{equation}

\noindent
Because $x$ was arbitrary, we have

\begin{equation}
    \brackets{\bigcup_{\alpha\in\mathcal{A}} A_{\alpha}}^{c}
        \subseteq \bigcap_{\alpha\in\mathcal{A}} A_{\alpha}^{c}.
    \label{eq:app_math_set_inequality_dml_1}
\end{equation}

We now suppose

\begin{equation}
    x\in\bigcap_{\alpha\in\mathcal{A}} A_{\alpha}^{c}.
\end{equation}

\noindent
This means that $x\in A_{\alpha}^{c}$ for all $\alpha\in\mathcal{A}$.
It follows that $x\notin A_{\alpha}$ for all $\alpha\in\mathcal{A}$.
Thus, we have

\begin{equation}
    x\notin\bigcup_{\alpha\in\mathcal{A}} A_{\alpha}.
\end{equation}

\noindent
This is equivalent to

\begin{equation}
    x\in\brackets{\bigcup_{\alpha\in\mathcal{A}} A_{\alpha}}^{c}.
\end{equation}

\noindent
Because $x$ was arbitrary, we have

\begin{equation}
    \bigcap_{\alpha\in\mathcal{A}} A_{\alpha}^{c} \subseteq
        \brackets{\bigcup_{\alpha\in\mathcal{A}} A_{\alpha}}^{c}.
    \label{eq:app_math_set_inequality_dml_2}
\end{equation}

Taking Eqs.~\eqref{eq:app_math_set_inequality_dml_1}
and \eqref{eq:app_math_set_inequality_dml_2}
along with Theorem~\ref{thm:set_equality} shows us that

\begin{equation}
    \brackets{\bigcup_{\alpha\in\mathcal{A}} A_{\alpha}}^{c}
        = \bigcap_{\alpha\in\mathcal{A}} A_{\alpha}^{c}.
\end{equation}

\noindent
This completes the proof of the first equality.
The second is left as an exercise to the reader.
\end{proof}

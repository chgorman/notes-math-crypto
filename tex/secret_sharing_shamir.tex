\section{Shamir's Secret Sharing}
\label{sec:ss_shamir}

\subsection{Setup}

We now discuss Shamir's Secret Sharing protocol~\cite{shamir1979share}.

We want a $\parens{t,n}$-sharing protocol,
where $n$ is the total number of participants
and we require $t+1$ secret shares to recover the original secret.
To that end, let $P_{i}$ be the $i$th participant
and let $\mathcal{P} = \braces{P_{i}}_{i=1}^{n}$ be the set
of all participants.
\emph{Note well:} this protocol requires \emph{trusted setup};
that is, whoever is producing the secret shares must be honest.

To do this, let $q\in\N$ be a prime sufficiently large.
Also, let $s\chooseRandom{}\F_{q}$ be the secret that we wish to distribute.
Additionally, we choose random coefficients
$c_{1}, c_{2}, \cdots, c_{t} \chooseRandom{} \F_{q}$;
we set $c_{0} = s$.
At this point, we have the secret polynomial

\begin{equation}
    p(x) = c_{0} + c_{1}x + c_{2}x^{2} + \cdots + c_{t}x^{t}.
\end{equation}

We use this secret polynomial to derive the secret shares.
We let $\parens{i, p(i)}$ be the share for participant $P_{i}$.
Because $\braces{c_{i}}_{i=0}^{n}$ are chosen uniformly at random in $\F_{q}$,
$p(i)$ will appear to be randomly distributed in $\F_{q}$.

\subsection{Secret Reconstruction}

We now show how to reconstruct the secret $s = c_{0}$.

We let $\mathcal{R}\subseteq\mathcal{P}$
be a valid subset of participants who are able to reconstruct the secret;
that is, we have $\abs{\mathcal{R}} = t+1$.
We now denote the secret share for participant $P_{k}$ as

\begin{equation}
    s_{k}\mathDef{} p(k).
\end{equation}

\noindent
This is meant to simplify the equations below.

We review some information related to \gls{lagrange interpolation}
from Chapter~\ref{sec:math_lagrange}.
The Lagrange interpolating polynomial for
$\braces{\parens{x_{k},y_{k}}}_{k=1}^{n}$ is

\begin{align}
    L(x) &\mathDef{} \sum_{k=1}^{n} y_{k}\ell_{k}(x) \nonumber\\
    \ell_{k}(x) &\mathDef{}
        \prod_{\substack{1\le j \le n \\ j\ne k}} \frac{x-x_{j}}{x_{k}-x_{j}}.
\end{align}

\noindent
These equations were previously presented in Eq.~\eqref{eq:math_lagrange_def}.
In this setting, our data is
$\braces{\parens{k, s_{k}}}_{P_{k}\in\mathcal{R}}$.
Thus, we have the following interpolating polynomial:

\begin{align}
    L(x) &= \sum_{P_{k}\in\mathcal{R}} s_{k}\ell_{k}(x) \nonumber\\
    \ell_{k}(x) &=
        \prod_{\substack{P_{j}\in\mathcal{R} \\ j\ne k}}
            \frac{x-j}{k-j} \nonumber\\
    &= \prod_{\substack{P_{j}\in\mathcal{R} \\ j\ne k}}
            \frac{j-x}{j-k}.
\end{align}

\noindent
In order to compute the secret $s$, we need to evaluate $L(0)$.
This gives us

\begin{align}
    s &= \sum_{P_{k}\in\mathcal{R}} s_{k} R_{k} \nonumber\\
    R_{k} &= \prod_{\substack{P_{j}\in\mathcal{R} \\ j\ne k}}
        \frac{j}{j-k}.
\end{align}

\subsection{Examples}

We work through some examples to understand the entire process.

\begin{example}[Alice, Bob, and Charlie share a secret]
\exampleCodeReference{examples/secret\_sharing/secret\_sharing\_1.py}

As mentioned previously, Alice, Bob, and Charlie formed a business
and want to share a secret key to their bank account whereby
2 out of 3 must sign off on any withdrawals.
Thus, they want a $\parens{1,3}$-threshold system.

We begin by choosing $q = 7919$.
We have the secret $s = 2310$ and polynomial

\begin{equation}
    p(x) = 2310 + 4673x.
\end{equation}

\noindent
In this case, we see Alice, Bob, and Charlie receive the shares

\begin{align}
    \text{Alice} &= \parens{1, p(1)} \nonumber\\
        &= \parens{1, 6983} \nonumber\\
    \text{Bob} &= \parens{2, p(2)} \nonumber\\
        &= \parens{2, 3737} \nonumber\\
    \text{Charlie} &= \parens{3, p(3)} \nonumber\\
        &= \parens{3, 491}.
\end{align}

We now look at all three possible share combinations:
Alice and Bob, Alice and Charlie, and Bob and Charlie.

\begin{itemize}
\item Alice and Bob:

In this case, we have that

\begin{align}
    s &= s_{1}R_{1} + s_{2}R_{2} \nonumber\\
    s_{1} &= 6983 \nonumber\\
    s_{2} &= 3737 \nonumber\\
    R_{1} &= \frac{2}{2-1} \mod 7919 \nonumber\\
    R_{2} &= \frac{1}{1-2} \mod 7919.
\end{align}

\noindent
We reduce these modulo $q$ to find

\begin{align}
    R_{1} &= \frac{2}{2-1} \mod 7919 \nonumber\\
        &= 2 \nonumber\\
    R_{2} &= \frac{1}{1-2} \mod 7919 \nonumber\\
        &= -1 \mod 7919 \nonumber\\
        &= 7918.
\end{align}

\noindent
This reduces to

\begin{align}
    s &= 6983\cdot2 + 3737\cdot7918 \mod 7919 \nonumber\\
        &= 2310.
\end{align}

\noindent
Thus, we have arrived at the \gls{shared secret} $s$.

\item Alice and Charlie:

In this case, we have that

\begin{align}
    s &= s_{1}R_{1} + s_{3}R_{3} \nonumber\\
    s_{1} &= 6983 \nonumber\\
    s_{3} &= 491 \nonumber\\
    R_{1} &= \frac{3}{3-1} \mod 7919 \nonumber\\
    R_{3} &= \frac{1}{1-3} \mod 7919.
\end{align}

\noindent
In this case, we actually have to do some operations modulo $q$.
We see that

\begin{align}
    R_{1} &= \frac{3}{2} \mod 7919 \nonumber\\
        &= 3961 \nonumber\\
    R_{3} &= -\frac{1}{2} \mod 7919 \nonumber\\
        &= 3959.
\end{align}

\noindent
This reduces to

\begin{align}
    s &= 6983\cdot3961 + 491\cdot3959 \mod 7919 \nonumber\\
        &= 2310.
\end{align}

\noindent
Thus, we have arrived at the \gls{shared secret} $s$.

\item Bob and Charlie:

In this case, we have that

\begin{align}
    s &= s_{2}R_{2} + s_{3}R_{3} \nonumber\\
    s_{2} &= 3737 \nonumber\\
    s_{3} &= 491 \nonumber\\
    R_{2} &= \frac{3}{3-2} \mod 7919 \nonumber\\
    R_{3} &= \frac{2}{2-3} \mod 7919.
\end{align}

\noindent
We see that

\begin{align}
    R_{2} &= 3 \nonumber\\
    R_{3} &= \frac{2}{2-3} \mod 7919 \nonumber\\
        &= -2 \mod 7919 \nonumber\\
        &= 7917.
\end{align}

\noindent
This reduces to

\begin{align}
    s &= 3737\cdot3 + 491\cdot7917 \mod 7919 \nonumber\\
        &= 2310.
\end{align}

\noindent
Thus, we have arrived at the \gls{shared secret} $s$.
\end{itemize}

In each case, every pair of participants are able to derive the
\gls{shared secret}.
\end{example}

\begin{example}[Alice, Bob, Charlie, and Dave share a secret]
\label{example:dkg_shamir_4}
\exampleCodeReference{examples/secret\_sharing/secret\_sharing\_2.py}

After making some good progress on their business,
Alice, Bob, and Charlie decided to add Dave as a business partner.
Again, with all treating each other as equals, they want
the ability for a strict majority of them to be required to derive
the secret key for the bank account.
Thus, they now need to create a new secret and use a $\parens{2,4}$
threshold system.

As before, they choose $q = 7919$.
This time, they have the secret $s = 1770$ and the secret polynomial

\begin{equation}
    p(x) = 1770 + 6540x + 2889x^{2}.
\end{equation}

\noindent
In this case, we see Alice, Bob, Charlie, and Dave receive the shares

\begin{align}
    \text{Alice} &= \parens{1, p(1)} \nonumber\\
        &= \parens{1, 3280} \nonumber\\
    \text{Bob} &= \parens{2, p(2)} \nonumber\\
        &= \parens{2, 2649} \nonumber\\
    \text{Charlie} &= \parens{3, p(3)} \nonumber\\
        &= \parens{3, 7796} \nonumber\\
    \text{Dave} &= \parens{4, p(4)} \nonumber\\
        &= \parens{4, 2883}.
\end{align}

We will look at two examples of deriving the \gls{shared secret}:
the triple Alice, Bob, and Dave as well as the triple Alice, Charlie, and Dave.

\begin{itemize}
\item Alice, Bob, and Dave:

In this case, we have that

\begin{align}
    s &= s_{1}R_{1} + s_{2}R_{2} + s_{4}R_{4} \nonumber\\
    s_{1} &= 3280 \nonumber\\
    s_{2} &= 2649 \nonumber\\
    s_{4} &= 2883 \nonumber\\
    R_{1} &= \frac{2}{2-1}\cdot\frac{4}{4-1} \mod 7919 \nonumber\\
    R_{2} &= \frac{1}{1-2}\cdot\frac{4}{4-2} \mod 7919 \nonumber\\
    R_{4} &= \frac{1}{1-4}\cdot\frac{2}{2-4} \mod 7919.
\end{align}

\noindent
Performing these operations we find

\begin{align}
    R_{1} &= \frac{2}{2-1}\cdot\frac{4}{4-1} \mod 7919 \nonumber\\
        &= \frac{8}{3} \mod 7919 \nonumber\\
        &= 5282 \nonumber\\
    R_{2} &= \frac{1}{1-2}\cdot\frac{4}{4-2} \mod 7919 \nonumber\\
        &= -2 \mod 7919 \nonumber\\
        &= 7917 \nonumber\\
    R_{4} &= \frac{1}{1-4}\cdot\frac{2}{2-4} \mod 7919 \nonumber\\
        &= \frac{1}{3} \mod 7919 \nonumber\\
        &= 2640.
\end{align}

\noindent
This reduces to

\begin{align}
    s &= 3280\cdot5282 + 2649\cdot7917 + 2883\cdot2640 \mod 7919
        \nonumber\\
    &= 1770.
\end{align}

\noindent
Thus, we have arrived at the \gls{shared secret} $s$.

\item Alice, Charlie, and Dave:

In this case, we have that

\begin{align}
    s &= s_{1}R_{1} + s_{3}R_{3} + s_{4}R_{4} \nonumber\\
    s_{1} &= 3280 \nonumber\\
    s_{3} &= 7796 \nonumber\\
    s_{4} &= 2883 \nonumber\\
    R_{1} &= \frac{3}{3-1}\cdot\frac{4}{4-1} \mod 7919 \nonumber\\
    R_{3} &= \frac{1}{1-3}\cdot\frac{4}{4-3} \mod 7919 \nonumber\\
    R_{4} &= \frac{1}{1-4}\cdot\frac{3}{3-4} \mod 7919.
\end{align}

\noindent
Performing these operations we find

\begin{align}
    R_{1} &= \frac{3}{3-1}\cdot\frac{4}{4-1} \mod 7919 \nonumber\\
        &= 2 \nonumber\\
    R_{3} &= \frac{1}{1-3}\cdot\frac{4}{4-3} \mod 7919 \nonumber\\
        &= -2 \mod 7919 \nonumber\\
        &= 7917 \nonumber\\
    R_{4} &= \frac{1}{1-4}\cdot\frac{3}{3-4} \mod 7919 \nonumber\\
        &= 1.
\end{align}

\noindent
This reduces to

\begin{align}
    s &= 3280\cdot2 + 7796\cdot7917 + 2883\cdot1 \mod 7919
        \nonumber\\
    &= 1770.
\end{align}

\noindent
Thus, we have arrived at the \gls{shared secret} $s$.
\end{itemize}

In both cases we were able to derive the \gls{shared secret} key.
\end{example}

\subsection{Discussion}
\label{ssec:ss_shamir_discussion}

This protocol works well but has some problems.
First, it requires \emph{trusted setup}; there has to be a trusted
party who determines the secret shares and distributes them
to each participant.
Additionally, we assume each participant \emph{correctly shares}
his portion of the secret.
Furthermore, once a participant has shared his secret,
there is nothing stopping the other participants from stealing it.

The required truthfulness of the dealer as well as the other participants
has lead to further development of secret sharing protocols
which are designed to reduce these requirements.
Even so, the protocols discussed below build upon the material here.

We have not explicitly mentioned where $q$ is used.
The size of $q$ defends the system against brute force attack.
In practice, $q$ needs to be large enough so that it is impractical
to guess the key.
Thus, choosing $q$ as a 256-bit prime number would be reasonable.
Of course, all of this depends on the desired level of security.

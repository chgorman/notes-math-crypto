\chapter{Mathematical Review: Groups, Rings, and Fields}
\chaptermark{Mathematical Review 2}
\label{chap:math_2}

We now build on our mathematical discussion from Chapter~\ref{chap:math_1}.
Here, we focus on the algebraic objects known as \glspl{group},
\glspl{ring}, and \glspl{field}.
These mathematical objects enable us to discuss the specifics
of \gls{publiccrypto}.

To assist in this discussion, we will first begin with intuition
and examples before giving the formal definition;
we include non-examples as well.

\section{Groups}
\label{sec:math_groups}

\subsection{Why do we care about Groups?}
\Glspl{group} arise in many different areas.
In cryptography, we will encounter them in the \gls{dhke}
and \glspl{signature}.
For \glspl{group} to be used in practice, we will need to encode them;
that is, we will need to know how to represent them on a computer.

\subsection{Intuition and Examples}
\emph{\Glspl{group}} are mathematical objects which we will
frequently encounter.
Informally, a \gls{group} is a generalization of the integers under addition,
so we start by looking at them first.

\begin{example}[Integers under addition]
The standard example of a \gls{group} is to consider the
integers $\Z$ under addition.

Given $a,b\in\Z$, we know that $a+b\in\Z$.
This means that the integers are \emph{closed} under addition:
adding two integers will always give us another integer.

Addition is also \emph{\gls{associative}}.
That is, for $a,b,c\in\Z$, we have

\begin{equation}
    \parens{a + b} + c = a + \parens{b + c}.
\end{equation}

\noindent
In this way, we do not have to worry about which way we add integers;
we will always get the same result regardless of order.

There is also a special integer $0\in\Z$.
It is special in that adding $0$ to $a$ gives us $a$:

\begin{equation}
    0 + a = a + 0 = a.
\end{equation}

\noindent
No other integer has this property.

We also know the (additive) inverse of $a$: $-a$.
This is because

\begin{equation}
    a + (-a) = 0.
\end{equation}

\noindent
It also turns out that the (additive) inverse is \emph{unique}.

We note that every integer can be written as
repeated additions of $1$.
If $a\ge0$, then we have

\begin{equation}
    a = \underbrace{1 + 1 + \cdots + 1}_{\text{$a$ times}} \quad a\ge 0.
    \label{eq:math_groups_integers_1_gen_pos}
\end{equation}

\noindent
For this to apply to negative numbers, we allow for adding $1$'s
additive inverse $-1$.
If $a<0$, then

\begin{equation}
    a = \underbrace{(-1) + (-1) + \cdots + (-1)}_{\text{$\abs{a}$ times}}
        \quad a < 0.
    \label{eq:math_groups_integers_1_gen_neg}
\end{equation}

\noindent
In this way, $1$ \emph{generates} the \gls{group} of integers under addition.

We also notice that addition is \emph{\gls{commutative}}:
given $a,b\in\Z$, we have

\begin{equation}
    a + b = b + a.
\end{equation}

\noindent
Although not all \glspl{group} have this property, the ones we care about do.
\end{example}

\begin{example}[Rationals under addition]
Similar to the integers under addition, the rational numbers $\Q$
under addition also form a \gls{group}.

Given $a,b\in\Q$ with $a = p/q$ and $b = r/s$,
we know

\begin{align}
    a + b &= \frac{p}{q} + \frac{r}{s} \nonumber\\
        &= \frac{ps + qr}{qs}\in\Q.
\end{align}

\noindent
This shows us that the rationals are closed under addition.

We know that addition of rationals is \gls{associative} and \gls{commutative}.
Furthermore, $0\in\Q$ is the identity element.
If $a = \frac{p}{q}$ as above, then the inverse is

\begin{equation}
    -a = \frac{-p}{q}
\end{equation}

\noindent
because

\begin{align}
    a + \parens{-a} &= \frac{p}{q} + \frac{-p}{q} \nonumber\\
        &= \frac{pq - pq}{q^{2}} \nonumber\\
        &= 0.
\end{align}

\noindent
This shows that the rationals under addition form a \gls{group}.
\end{example}

\begin{example}[Integers under modular addition]
We now look at another standard example: integers under modular arithmetic.
Let $n > 1$.
Then

\begin{equation}
    \Z_{n} = \braces{0, 1, \cdots, n-1}.
\end{equation}

\noindent
For $a,b\in\Z_{n}$, we know there is some $c\in\Z_{n}$ such that

\begin{equation}
    a + b = c \mod n.
\end{equation}

\noindent
This shows us that $\Z_{n}$ is closed under addition.

We know $0\in\Z_{n}$ and we always have

\begin{equation}
    0 + a = a + 0 = a
\end{equation}

\noindent
for $a\in\Z_{n}$.
Thus, $0$ is the additive identity.

Given $a\in\Z_{n}\setminus\braces{0}$, we know

\begin{equation}
    a + \parens{n-a} = 0 \mod n,
\end{equation}

\noindent
Thus, the additive inverse of $a$ is $n-a$;
$0$ is its own additive inverse.
Therefore, we see that $\parens{\Z_{n},+}$ is a \gls{group}.
\end{example}

\begin{example}[Positive Rationals under multiplication]
Similar to the integers under addition, the positive rational
numbers $\Q^{+}$ under multiplication form a \gls{group}.
In this case, the multiplicative identity is $1$.
\end{example}

\begin{example}[Nonexample: Natural numbers under addition]
We know that the natural numbers under addition are
very similar to the integers under addition.
We know that we can always add two natural numbers,
so that $a,b\in\N$ implies $a+b\in\N$;
thus, the naturals are closed under addition.
Addition is also \gls{associative} and \gls{commutative}.

We also have the identity $0\in\N$, so

\begin{equation}
    a + 0 = a
\end{equation}

\noindent
for all $a\in\N$.

Unfortunately, we do not have additive inverses;
this is because the natural numbers do not include the negative integers.
That is, for $a\in\N\setminus\braces{0}$, there is no $b\in\N$ such that

\begin{equation}
    a + b = 0.
\end{equation}

\noindent
Thus, $\parens{\N,+}$ is \emph{not} a \gls{group}.
\end{example}

\subsection{Formal Definition}

\begin{defn}[Group]
A \gls{group} is a \gls{set} $G$ together with a binary operation $\cdot$
such that, given $a,b\in G$, $a\cdot b\in G$;
that is, $G$ is closed under the binary operation $\cdot$.

The binary operation also satisfies the following properties:

\begin{itemize}
\item Associativity: for all $a,b,c\in G$, we have

\begin{equation}
    \parens{a\cdot b}\cdot c = a\cdot \parens{b\cdot c}.
\end{equation}

\item Identity element: there is an element $e\in G$ such that
    for all $a\in G$,

\begin{equation}
    a\cdot e = e\cdot a = a.
\end{equation}

\item Inverses: for every $a\in G$ there is a $b\in G$ such that
    
\begin{equation}
    a\cdot b = b\cdot a = e.
\end{equation}

\noindent
We henceforth denote the inverse of $a$ as $a^{-1}$.
\end{itemize}
\end{defn}

Formally, we write that $\parens{G,\cdot}$ is a \gls{group}.
Informally, we write that $G$ is a \gls{group} when the operation is understood.
Additionally, we frequently write $ab$ for $a\cdot b$.

\subsection{Continued Discussion}
Although the generic group operation was listed as ``multiplication'',
we could very well have used addition.
We will sometimes refer to some \glspl{group} as ``multiplicative groups''
while others as ``additive groups'';
this just means we use the multiplication sign $\cdot$
or the addition sign $+$ to denote the group operation.
Nothing changes, as we can use both symbols,
but there are conventions.
In additive groups, the identity element is usually denoted $0$;
in multiplicative groups, the identity element is usually denoted $1$.

Because $\parens{\Z,+}$ is a \gls{group}, we know that \glspl{group} can have
an infinite number of elements.
In what follows, we will be particularly interested
in \emph{\glspl{finite group}}:
\glspl{group} that have a finite number of elements.
We let $\abs{G}$ denote the order of the \gls{group}
(the number of elements in $G$).

At times, we may drop explicit reference to the operation if it is understood.
For instance, we may refer to $\Z$ as a \gls{group},
when more formally we should say $\parens{\Z,+}$ is a \gls{group}
in order to reference both the \emph{\gls{set}} (the integers $\Z$)
and the \emph{operation} (integer addition).
Even so, this is long, tedious, and painful, so we will,
at times, forego the formalities when the operation is clear
from the context.

\subsection{More Definitions}
Let $G$ be a \gls{group} and $H \subseteq G$;
more explicitly, $\parens{G,\cdot}$ is a \gls{group} and $H$ is a subset of $G$.
We say that $H$ is a \emph{\gls{subgroup}} of $G$, written as $H \le G$,
if $H$ with the group operation inherited from $G$ is also a \gls{group}.
Every \gls{group} $G$ has at least two \glspl{subgroup}:
$G$ (the entire group) and $\braces{e}$
(the group that is just the identity element $e$).
A \gls{subgroup} is called a \emph{proper subgroup} when
$H\le G$ and $H\ne G$.
If $H$ is a proper subgroup, then we may write $H < G$.
The \gls{group} $\braces{e}$ is called the \emph{trivial} subgroup.

\begin{example}[Subgroup Example: $2\Z \le \Z$]
We let

\begin{equation}
    2\Z \mathDef{} \braces{\ldots, -4, -2, 0, 2, 4, \ldots};
\end{equation}

\noindent
that is, $2\Z$ is the set of even integers.
Naturally, we have $2\Z\subseteq\Z$.
For every $a\in2\Z$ there is some $b\in\Z$ so that $a = 2b$.

Given $x,y\in2\Z$, we set $x = 2m$ and $y = 2n$ for $m,n\in\Z$.
We then see

\begin{align}
    x + y &= 2m + 2n \nonumber\\
        &= 2\parens{m+n}\in2\Z.
\end{align}

\noindent
Thus, $2\Z$ is closed under addition.
Because addition on $\Z$ is \gls{associative},
addition on $2\Z$ is also \gls{associative}.

Because $0\in\Z$ and

\begin{equation}
    0 + x = x\in2\Z,
\end{equation}

\noindent
we still have the same identity element $0\in2\Z$.

We let $x\in2\Z$ with $x = 2m$ for $m\in\Z$.
If $y = -2m$, then $y\in\Z$ and

\begin{align}
    x + y &= 2m + \parens{-2m} \nonumber\\
        &= 0\in2\Z.
\end{align}

\noindent
Thus, we see that every element in $2\Z$ has an additive inverse
within $2\Z$.
We have just shown that $\parens{2\Z,+}$ is a \gls{group};
because $2\Z\subseteq\Z$, $2\Z$ is a \gls{subgroup} of $\Z$.

In general, for $n\in\N\setminus\braces{0}$ we let

\begin{equation}
    n\Z \mathDef{} \braces{\ldots, -2n, -n, 0, n, 2n, \ldots}.
\end{equation}

\noindent
Then $n\Z\le\Z$.
This is a \emph{proper} subgroup when $n\ge2$.
\end{example}

For a multiplicative group $\parens{G,\cdot}$ and $g\in G$, we define

\begin{equation}
    \angles{g} \mathDef{} \braces{g^{n} \mid n\in\Z}.
\end{equation}

\noindent
In this case, we see that $\angles{g}$ contains all group elements
of the form $g^{n}$ for $n\in\Z$.
This is similar to Eqs.~\eqref{eq:math_groups_integers_1_gen_pos}
and \eqref{eq:math_groups_integers_1_gen_neg}.
Here, we have

\begin{equation}
    g^{n} = \underbrace{g \cdot g \cdots g}_{\text{$n$ times}} \quad n\ge 0
\end{equation}

\noindent
and

\begin{equation}
    g^{n} = \underbrace{g^{-1} \cdot g^{-1} \cdots g^{-1}}_{\text{$\abs{n}$
        times}} \quad n < 0.
\end{equation}

\noindent
In this case, $\angles{g}$ is the \gls{subgroup} of $G$ which consists
of products of $g$ or $g^{-1}$;
this allows us to write $\angles{g}\le G$.
We say $G$ is a \emph{\gls{cyclic group}} if there is a $g\in G$ such that
for all $h\in G$ there is an $n\in\Z$ such that $h = g^{n}$.
In this case, we have $G = \angles{g}$ and 
say that $g$ is a \emph{generator} of $G$.
\Glspl{cyclic group} are important in cryptography because
these \glspl{group} are used in the \gls{dhke};
they are also used to construct \glspl{signature}.

\begin{example}[Cyclic Group: $\Z$]
As we saw previously, $1$ is a generator of the integers $\Z$ under addition.
Thus, we can write $\angles{1} = \Z$.
We note that $-1$ is also a generator, so $\angles{-1} = \Z$.
\end{example}

\begin{example}[Non-cyclic Group: $\Q$]
We can consider the rational numbers $\Q$ under addition.
For any $a \in \Q\setminus\braces{0}$, we know that
and $\frac{a}{2}\in\Q\setminus\braces{0}$.
Even so, we see that $\frac{a}{2} \not\in \angles{a}$.
Thus, $\Q$ is not a \gls{cyclic group}.
\end{example}

In all of the groups we have looked up to this point,
all of our group operations satisfy $a\cdot b = b\cdot a$;
that is, the group operation is \emph{\gls{commutative}}.
We say that $G$ is an \emph{\gls{abelian group}} if for all $a,b\in G$
we have $a\cdot b = b\cdot a$.
This need not always be the case, but we will only consider
\glspl{abelian group} here because the \glspl{group} which normally arise
in cryptography are abelian.

We note that addition and multiplication are familiar
\gls{commutative} operations;
we always have $a+b = b+a$ and $a\cdot b = b\cdot a$ for real numbers.
We know that subtraction and division are \emph{not} \gls{commutative};
in general, we have $a-b \ne b-a$ and $\frac{a}{b} \ne \frac{b}{a}$.

We will usually denote $a\cdot b$ as $ab$ from now on
when there is no cause for confusion.

\subsection{Encoding Groups}

When working with the group $\Z_{n}$,
we can encode $k\in\Z_{n}$ by its binary representation.
Other groups may have different representations.

\section{Rings}
\label{sec:math_rings}

\subsection{Why do we care about Rings?}
\Glspl{ring} arise in many different areas.
We use \glspl{ring} to ease our way into the discussion about
\glspl{field} in Chapter~\ref{sec:math_fields}.

\subsection{Intuition and Examples}
\emph{\Glspl{ring}} are mathematical objects which we will sometimes encounter.
They are a generalization of the integers when we look
at both addition \emph{and} multiplication of integers.
We begin with some examples.

\begin{example}[Integers under addition and multiplication]
We know that we can add and multiply integers and remain
in the set of integers.
In this way, the integers are closed under addition
and closed under multiplication.
We also note that both addition and multiplication are \gls{associative}.
We know that both addition is \gls{commutative};
multiplication is also \gls{commutative},
but here we emphasize the \gls{commutative} nature of addition.

In the case of addition, we know that $0\in\Z$ is the additive inverse:
for every $a\in\Z$, we have

\begin{equation}
    0 + a = a + 0 = a.
\end{equation}

\noindent
Similarly, we have that $1\in\Z$ is the multiplicative inverse:
we have

\begin{equation}
    1\cdot a = a\cdot1 = a
\end{equation}

\noindent
for all $a\in\Z$.

We also know that addition and multiplication follow certain rules.
In particular, given $a,b,c\in\Z$, we have

\begin{equation}
    a\cdot\parens{b+c} = \parens{a\cdot b} + \parens{a\cdot c};
\end{equation}

\noindent
that is, we can \emph{distribute} multiplication over addition.
We also have

\begin{equation}
    \parens{a+b}\cdot c = \parens{a\cdot c} + \parens{b\cdot c}.
\end{equation}
\end{example}

\begin{example}[Integers modulo $n$]
We now look at modular arithmetic.
To be concrete, we look at $\Z_{6}$.
This satisfies all of the properties of the integers under addition
and multiplication.

One interesting property of $\Z_{6}$ but not $\Z$ is that $2,3\in\Z_{6}$
with $2\ne0$ and $3\ne0$ but

\begin{equation}
    2\cdot 3 = 0 \mod 6.
\end{equation}

\noindent
This is an example where two nonzero elements may be multiplied
together to equal zero.
Thus, we can see that there is something distinctly different
between $\Z$ and $\Z_{6}$.
\end{example}

\subsection{Formal Definition}

\begin{defn}[Ring]
A \gls{ring} is a \gls{set} $R$ together with two binary operations addition $+$
and multiplication $\cdot$.
First, $R$ is closed under addition and multiplication.
Furthermore, we have the following properties:

\begin{itemize}
\item $\parens{R,+}$ is an \gls{abelian group} with additive identity $0$.
    This implies that addition is \gls{associative}.
\item Multiplication is \gls{associative} on $R$; that is,
    for all $a,b,c \in R$, we have

\begin{equation}
    \parens{a\cdot b}\cdot c = a\cdot\parens{b\cdot c}.
\end{equation}
\item There is multiplicative identity $1\in R$; that is, 
    for all $a\in R$, we have

\begin{equation}
    1\cdot a = a\cdot 1 = a.
\end{equation}

\noindent
Furthermore, $0\ne 1$.

\item We have the following distribution laws between multiplication
    and addition.
    For all $a,b,c\in R$, we have

\begin{align}
    a\cdot\parens{b + c} &= \parens{a\cdot b} + \parens{a\cdot c}
        \nonumber\\
    \parens{b + c}\cdot a &= \parens{b\cdot a} + \parens{c\cdot a}
\end{align}
\end{itemize}

\noindent
We formally write $\parens{R,+,\cdot}$ is a ring.
\end{defn}

The above definition holds for all rings.
A \emph{\gls{commutative ring}} is one where multiplication
is \gls{commutative};
that is $a\cdot b = b\cdot a$ for all $a,b\in R$.
We will focus on \glspl{commutative ring} here because those will arise
in our work moving forward.
Thus, our distribution law reduces to 

\begin{equation}
    a\cdot\parens{b + c} = \parens{a\cdot b} + \parens{a\cdot c}.
\end{equation}

\subsection{Continued Discussion}

We say that  $a\in R\setminus\braces{0}$ is a \emph{zero divisor}
if there exists $b\in R\setminus\braces{0}$ such that $ab = 0$.

\begin{example}[Example of Zero Divisors]
We continue to look at the \gls{ring} $\Z_{6}$.
We have $2,3\in\Z_{6}$ with $2\ne0$ and $3\ne0$, yet we know

\begin{equation}
    2\cdot 3 = 6 \equiv 0 \mod 6.
\end{equation}

\noindent
This implies that $2$ and $3$ are zero divisors in $\Z_{6}$.

More generally, let $n = pq$ for distinct primes $p$ and $q$.
Then $p,q\in\Z_{n}$ and we know

\begin{equation}
    p\cdot q = n \equiv 0 \mod n.
\end{equation}

\noindent
This shows that $p$ and $q$ are zero divisors in $\Z_{n}$.
\end{example}

\begin{example}[Non-example of Zero Divisors]
While we may not have used this language before,
$\Z$ has no zero divisors.
We know (although we have not shown) that for $a,b\in\Z$ with
$ab = 0$ implies that $a=0$ or $b=0$.

This shows that the \gls{ring} $\parens{\Z,+,\cdot}$
is very different from the \glspl{ring} $\parens{\Z_{n},+,\cdot}$
when $n$ is composite.
\end{example}

\subsection{Encoding Rings}

When working with the \gls{ring} $\Z_{n}$, we can encode $k\in\Z_{n}$
by its binary representation.
Other rings have different representations.

\section{Fields}
\label{sec:math_fields}

\subsection{Why do we care about Fields?}
\Glspl{field} arise in many different areas.
In cryptography, we use \glspl{field} in \gls{dhke} and \glspl{signature}.
Additionally, \glspl{field} are used when working with \glspl{elliptic curve}.

\subsection{Intuition and Examples}
\emph{\Gls{field}} are common mathematical objects.
All \glspl{field} are \glspl{ring} but have additional properties.
Informally, a \gls{field} is a generalization of the rational numbers
with addition and multiplication.
In particular, every nonzero element has a multiplicative inverse.
We begin with some examples.

\begin{example}[The Rational Numbers]
The rational numbers are a standard example of a \gls{field}.
For $a\in\Q\setminus\braces{0}$, we can write

\begin{equation}
    a = \frac{m}{n}
\end{equation}

\noindent
for $m,n\in\Z\setminus\braces{0}$.
Then $\frac{n}{m}\in\Q$ and

\begin{equation}
    \frac{m}{n} \cdot \frac{n}{m} = 1.
\end{equation}

\noindent
Thus, $a$ has multiplicative inverse.
The rational numbers $\Q$ have an infinite number of elements.
\end{example}

\begin{example}[The Field $(\Z_{11},+,\cdot)$]
\exampleCodeReference{examples/math\_review/finite\_field\_inverses.py}

Let us look at $\Z_{11}$.
We have the following multiplication facts:

\begin{align}
    1\cdot 1 &= 1 \mod 11
        &
    6\cdot 2 &= 1 \mod 11 \nonumber\\
    2\cdot 6 &= 1 \mod 11
        &
    7\cdot 8 &= 1 \mod 11 \nonumber\\
    3\cdot 4 &= 1 \mod 11
        &
    8\cdot 7 &= 1 \mod 11 \nonumber\\
    4\cdot 3 &= 1 \mod 11
        &
    9\cdot 5 &= 1 \mod 11 \nonumber\\
    5\cdot 9 &= 1 \mod 11
        &
    10\cdot 10 &= 1 \mod 11.
\end{align}

\noindent
This shows that every nonzero element in $\Z_{11}$ has a multiplicative
inverse.
This, along with the other \gls{ring} properties, makes $\Z_{11}$ a \gls{field}.
In this case, we write it as $\F_{11}$ to emphasize the \gls{field} nature.
The \gls{field} $\F_{11}$ has a finite number of elements.
\end{example}

\subsection{Formal Definition}
\begin{defn}[Field]
A \gls{field} is a \gls{set} $F$ together with two binary operations
addition $+$ and multiplication $\cdot$ which have the following properties:

\begin{itemize}
\item $\parens{F,+}$ is an \gls{abelian group} with additive identity $0$.

\item $\parens{F^{*},\cdot}$ is an \gls{abelian group} with multiplicative
    identity $1$.
    Here, $F^{*} \mathDef{} F\setminus\braces{0}$.

\item We have $0\ne 1$.

\item We have the following distribution law between multiplication
    and addition.
    For all $a,b,c\in F$, we have

\begin{equation}
    a\cdot\parens{b + c} = \parens{a\cdot b} + \parens{a\cdot c}.
\end{equation}
\end{itemize}

\noindent
We formally say that $\parens{F,+,\cdot}$ is a \gls{field}.
\end{defn}

\subsection{Continued Discussion}

\subsubsection{Difference Between Rings and Fields}

We note that a \gls{field} is a \gls{commutative ring} with no zero divisors
and where every nonzero element has a multiplicative inverse.

\subsubsection{Finite Fields}

While the \glspl{field} of the rationals $\Q$, the reals $\R$,
and complex numbers $\C$ are more familiar,
in cryptography we will be particularly interested
in \emph{\glspl{finite field}}.
As one may guess, a \gls{field} $\F$ is \emph{finite} if the number of
elements is finite; that is, $\abs{\F}<\infty$.

The \glspl{finite field} we will focus on are $\F_{p}$;
here $p$ is a prime number.
We will look at more examples of these \glspl{field} below.
There are additional types of \glspl{finite field},
but the \glspl{finite field} $\F_{p}$ are our primary focus.
Additional material on \glspl{field} may be found in
Appendix~\ref{app:math_finite_fields}.

\subsection{More Examples}

\begin{example}[Integers modulo a prime]
We now show that $\parens{\F_{p},+,\cdot}$ is a \gls{field},
where

\begin{equation}
    \F_{p} \mathDef{} \braces{0, 1, 2, \ldots, p-1}
\end{equation}

\noindent
and $p$ is prime.
We let

\begin{equation}
    \F_{p}^{*} \mathDef{} \F_{p}\setminus\braces{0},
\end{equation}

\noindent
so $\F_{p}^{*}$ are the nonzero elements of $\F_{p}$.

Let $a\in\F_{p}^{*}$ for prime $p>2$.
Because $p$ is prime and $1\le a\le p-1$,
$\gcd(a,p) = 1$.
Thus, there exists $x$ and $y$ such that

\begin{equation}
    ax + py = 1.
\end{equation}

\noindent
If we look at the above modulo $p$, we see

\begin{equation}
    ax = 1 \mod p.
\end{equation}

\noindent
Therefore, $a$ has a multiplicative inverse $x\mod p$.
This holds for all $a\in\F_{p}^{*}$, so $\parens{\F_{p},+,\cdot}$
is a \gls{field}.
\end{example}

\begin{example}[More examples with $\F_{p}$]
There are some unusual properties that we will discuss about $\F_{p}$.
Throughout this example, all arithmetic will be performed modulo $p$.

For all $x\in\F_{p}$, we have

\begin{equation}
    p\cdot x = 0.
\end{equation}

\noindent
By this, we mean

\begin{equation}
    \underbrace{x + x + \cdots + x}_{\text{$p$ times}} = 0.
\end{equation}

\noindent
In particular, we have

\begin{equation}
    p\cdot 1 = 0;
\end{equation}

\noindent
that is,

\begin{equation}
    \underbrace{1 + 1 + \cdots + 1}_{\text{$p$ times}} = 0.
\end{equation}

\noindent
This is not the case in the more familiar \glspl{field}
of $\Q$, $\R$, and $\C$.
In particular,

\begin{equation}
    \underbrace{1 + 1 + \cdots + 1}_{\text{$n$ times}} = n
\end{equation}

\noindent
for every $n\in\N$.
It is not possible to add $1$ to itself and arrive at $0$.

Additionally, for $x,y\in\F_{p}$, we have

\begin{equation}
    \parens{x+y}^{p} = x^{p} + y^{p}.
\end{equation}

\noindent
This follows from the previous property and the
\href{https://en.wikipedia.org/wiki/Binomial_theorem}{Binomial Theorem}.
Although this is fascinating, we will not discuss it further.
\end{example}

\begin{example}[Even more examples with $\F_{p}$]
\label{example:math_field_p_more}
\exampleCodeReference{examples/math\_review/finite\_field\_powers.py}

We now fix a value of $p$ in order to work on some examples.
We let

\begin{align}
    p &= 65537 \nonumber\\
        &= 2^{16} + 1.
\end{align}

Addition modulo $p$ is well understood.
We will take some time to look at examples of multiplication.
In particular, we will look at exponentiation.

We see

\begin{align}
    3^{0}  &= 1
        &
    3^{32}  &= 61869 \nonumber\\
    3^{1}  &= 3
        &
    3^{64}  &= 19139 \nonumber\\
    3^{2}  &= 9
        &
    3^{128}  &= 15028 \nonumber\\
    3^{4}  &= 81
        &
    3^{256}  &= 282 \nonumber\\
    3^{8}  &= 6561
        &
    3^{512}  &= 13987 \nonumber\\
    3^{16} &= 54449
        &
    3^{1024}  &= 8224.
\end{align}

\noindent
Once we reach a large enough exponent, say 16,
there does not appear to be any relationship between the exponent
$x$ and the resulting number $3^{x}\mod p$.

To get a \gls{group} from the \gls{field} $\F_{p}$,
we can look at the \gls{subgroup}
generated by 3 under multiplication;
that is, we can look at

\begin{equation}
    H \mathDef{} \braces{3^{x} \mod p \mid x\in\Z}.
\end{equation}

\noindent
Then $H\le \F_{p}^{*}$; that is, $H$ is a \gls{subgroup} of the \gls{group}
$\parens{\F_{p}^{*},\cdot}$.
In general, if we choose prime $p$ large enough,
we can use $H$ as a \gls{group} for the \gls{dhke};
this will be discussed in Chapter~\ref{sec:public_diffie_hellman}.
\end{example}

\subsection{Encoding Fields}

When working with the \gls{field} $\F_{p}$, we can encode $k\in\F_{p}$
by its binary representation.
Other \glspl{field} have different representations.




\section{Concluding Discussion}
It would be a valid question to ask if everything discussed here
is absolutely necessary to begin to understand cryptography.
The answer is \emph{yes} if one is really interested in getting
more than just a superficial view of \gls{publiccrypto}.
This is because \gls{publiccrypto} looks to work in \glspl{group}
where certain problems are deemed computationally infeasible.
We will discuss computational infeasibility more in Chapter~\ref{chap:hardness}.

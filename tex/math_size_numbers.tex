\section{Size of Numbers}

In \gls{publiccrypto} (which we discuss in Chapter~\ref{chap:public}),
we will need numbers to be of various magnitudes so that certain
mathematical problems are deemed sufficiently difficult;
the particular sizes are discussed in Chapter~\ref{chap:hardness}
when we talk about computational hardness assumptions.
We frequently represent integers in binary expansion,
so we use that to describe their sizes here.

Given $a\in\N$, we say that $a$ is a $k$-bit number if
$2^{k-1} \le a < 2^{k}$;
that is, the largest set bit in $a$'s binary expansion
is in the $k$th position.

\begin{example}[Size of Numbers]
\exampleCodeReference{examples/math\_review/size\_of\_numbers.py}

Here are some good order-of-magnitude estimates to keep in mind:

\begin{align}
    2^{32}   &\simeq 4\cdot10^{9}   \nonumber\\
    2^{64}   &\simeq 2\cdot10^{19}  \nonumber\\
    2^{128}  &\simeq 3\cdot10^{38}  \nonumber\\
    2^{256}  &\simeq 1\cdot10^{77}  \nonumber\\
    2^{512}  &\simeq 1\cdot10^{154} \nonumber\\
    2^{1024} &\simeq 2\cdot10^{308}.
\end{align}

\noindent
For reference, we note that the total number of particles
in the known universe is estimated to be $10^{80}$.
Additionally, a standard year has $31536000 \simeq 2^{25}$ seconds in it.
\end{example}

When working with cryptography in practice,
we may seek a certain level of security.
For instance, we may want ``128-bit security''.
This means that we want the most efficient algorithm for breaking
the underlying cryptosystem to require \emph{at least} $2^{128}$ operations.
As we can see, this is a large number.
Analogously, having $k$-bit security amounts to requiring
the most efficient algorithm must perform at least $2^{k}$ operations.

\begin{example}[How large is $2^{128}$?]
Let us suppose that we have $P$ processors
performing $F$ operations per second and run those processors for $S$ seconds.
The total number of operations $T$ that will be performed
in that amount of time will be

\begin{equation}
    T = P\cdot F\cdot S \quad\text{operations.}
\end{equation}

We now put in some concrete numbers.
Assume $P = 2^{30}$ processors (approximately one billion)
performing $F = 2^{30}$ operations per second (1 GHz)
for one year $S = 2^{25}$.
Then we have

\begin{align}
    T &= 2^{30}\cdot 2^{30} \cdot 2^{25} \nonumber\\
        &= 2^{85} \quad\text{operations.}
\end{align}

\noindent
At this rate, to perform $2^{128}$ operations would require
$2^{128-85} = 2^{43} \simeq 9\cdot10^{12}$ years.
It would take almost 10 trillion years at current capabilities
for perform $2^{128}$ operations.

While we should never try to estimate how computing will
advance in the future, it is clear that performing
$2^{128}$ operations is a daunting task.
Presumably, this is infeasible for at least the near future.
\end{example}

\subsection*{Converting between Powers of 2 and Powers of 10}

We have the following useful approximations:

\begin{align}
    10   &\simeq 2^{3} \nonumber\\
    100  &\simeq 2^{7} \nonumber\\
    1000 &\simeq 2^{10}.
\end{align}

\noindent
These approximations are slightly better:

\begin{align}
    10   &\simeq 2^{3.3} \nonumber\\
    100  &\simeq 2^{6.6} \nonumber\\
    1000 &\simeq 2^{10}.
\end{align}

\noindent
Even so, these approximations are probably not needed in practice,
especially if we just care about order-of-magnitude \emph{estimates}.

Based on $10^{3} = 1000 \simeq 1024 = 2^{10}$, we have the following:

\begin{align}
    2^{10} &\simeq 10^{3} \nonumber\\
    2^{20} &\simeq 10^{6} \nonumber\\
    2^{30} &\simeq 10^{9}.
\end{align}

\noindent
In general, we have

\begin{equation}
    2^{10k} \simeq 10^{3k}.
\end{equation}

\begin{example}[Power Conversions]
\exampleCodeReference{examples/math\_review/power\_conversion.py}

We will now work through some conversions.
We see

\begin{align}
    2^{86} &= 2^{6 + 10\cdot8} \nonumber\\
        &= 2^{6}\cdot2^{10\cdot8} \nonumber\\
        &\simeq 100\cdot10^{3\cdot8} \nonumber\\
        &= 10^{26}.
\end{align}

\noindent
A more exact approximation is

\begin{equation}
    2^{86} \simeq 7.7\cdot10^{25}.
\end{equation}

\noindent
Thus, $10^{26}$ is a good order-of-magnitude estimate.

Similarly,

\begin{align}
    10^{86} &= 10^{2 + 3\cdot28} \nonumber\\
        &= 100\cdot10^{3\cdot28} \nonumber\\
        &\simeq 2^{7}\cdot 2^{280} \nonumber\\
        &= 2^{287}.
\end{align}

\noindent
We see that

\begin{equation}
    \log_{2}(10^{86}) \simeq 285.7.
\end{equation}

\noindent
Thus, our approximation was not too far off:
we are within a factor of 4.

If instead we use the more accurate approximation $100 \simeq 2^{6.6}$,
then we have

\begin{equation}
    10^{86} \simeq 2^{286.6}.
\end{equation}

\noindent
This is within a factor of 2.
\end{example}

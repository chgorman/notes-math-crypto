% Glossary
%%%%%%%%%%%%%%%%%%%%%%%%%%%%%%%%%%%%%%%%%%%%%%%%%%%%%%%%%%%%%%%%%%%%%%%%
%%% Cryptography

\newglossaryentry{salt}
{
    name={salt},
    description={A random value used as additional data when working
        with \glsfirstplural{hash function}.
        When used within a \gls{hash function}, a salt ensures
        domain separation.
        Compare with \gls{nonce} and \gls{initialization vector}.},
}

\newglossaryentry{nonce}
{
    name={nonce},
    description={A \textbf{n}umber used only \textbf{once}.
        Compare with \gls{salt} and \gls{initialization vector}.},
}

\newglossaryentry{initialization vector}
{
    name={initialization vector},
    description={A bit string used to initialize an
        \gls{encryption scheme}.
        Compare with \gls{nonce} and \gls{salt}.},
}

\newglossaryentry{perfect security}
{
    name={perfect security},
    description={An \gls{encryption scheme} has \emph{perfect security}
        if it is not possible
        to break even with \emph{unbounded} computation.
        The \gls{otp} is the only perfectly secure \gls{encryption scheme}.
        Also called \emph{information-theoretic security}.},
}

\newglossaryentry{encryption scheme}
{
    name={encryption scheme},
    description={A general method for converting a message (\emph{plaintext})
        into random-looking data (\emph{ciphertext}).
        Alice and Bob use an encryption scheme to communicate securely
        over an \gls{insecure channel}.},
}

\newglossaryentry{public key encryption}
{
    name={public key encryption},
    description={An \gls{encryption scheme} which uses public keys
        and private keys for secure communication; part of \gls{publiccrypto}.},
}

\newglossaryentry{symmetric key encryption}
{
    name={symmetric key encryption},
    description={An \gls{encryption scheme} which uses one secret key
        for secure communication; part of \gls{symmetriccrypto}.
        Examples include the \gls{otp}, \glspl{stream cipher},
        and \glspl{block cipher}.},
}

\newglossaryentry{shared secret}
{
    name={shared secret},
    description={A secret that is shared between individuals.},
}

\newglossaryentry{distributed key generation}
{
    name={distributed key generation},
    description={A method for constructing a secret key between multiple
        participants and have it distributed between them.
        After the secret key is constructed, it is possible construct
        valid \glspl{signature} for the entire group
        (called \emph{group signatures}).},
}

\newglossaryentry{signature}
{
    name={digital signature},
    description={A method for authenticating data in \gls{publiccrypto}.
        Standard examples include DSA, ECDSA, and EdDSA.
        Digital signatures allow for nonrepudiation.},
}

\newglossaryentry{secure channel}
{
    name={secure channel},
    description={A method for communication without concern for adversaries.
        For instance, when Alice and Bob communicate over a secure channel,
        Eve will not learn what they discuss.
        There is assurance of privacy.},
}

\newglossaryentry{insecure channel}
{
    name={insecure channel},
    description={A method for communication where no secrecy is assured.
        For instance, when Alice and Bob communicate over an insecure channel,
        they assume that Eve will learn everything they discuss.
        There is no privacy on insecure channels.},
}

\newglossaryentry{hash function}
{
    name={hash function},
    description={A method for producing a constant-length digital identifier
        for data.
        The full name is \textbf{cryptographic hash function}.},
    first={cryptographic hash function},
}

\newglossaryentry{random oracle}
{
    name={random oracle},
    description={The ideal \glsfirst{hash function}
        with all of its desired properties.
        It is a useful theoretical construct;
        in practice, it would be instantiated with a
        \glsfirst{hash function}.},
}

\newglossaryentry{kdf}
{
    name={key derivation function},
    description={A method for converting nonuniform randomness into
        uniform randomness.
        Also used to store passwords.},
}

\newglossaryentry{hkdf}
{
    name={HMAC-based Key Derivation Function},
    description={A general method for constructing
        a \gls{kdf} from a \gls{hash function}.},
}

\newglossaryentry{mgf}
{
    name={mask generation function},
    description={A deterministic method for producing a digital identifier
        of variable length;
        similar to a \glsfirst{hash function}.
        It specifies one way to enable a \gls{hash function}
        to have a variable output length.
        Compare with \gls{xof}.},
}

\newglossaryentry{xof}
{
    name={extendable output function},
    description={A deterministic method for producing a digital identifier
        of variable length;
        similar to a \glsfirst{hash function}.
        Compare with \gls{mgf}.},
}

\newglossaryentry{csprng}
{
    name={pseudorandom number generator},
    description={A method for producing cryptographically-secure
        random numbers derived from a seed.
        The full name is
        \textbf{cryptographically-secure pseudorandom number generator}.
        One example is Fortuna.},
    first={cryptographically-secure pseudorandom number generator},
}

\newglossaryentry{otp}
{
    name={one-time pad},
    description={The only \gls{encryption scheme} with \gls{perfect security}.
        The challenge is that the key must be truly random
        and as long as the message.},
}

\newglossaryentry{stream cipher}
{
    name={stream cipher},
    description={An \gls{encryption scheme} which takes a secret key
        and stretches it into a stream of randomness.}
}

\newglossaryentry{block cipher}
{
    name={block cipher},
    description={An \gls{encryption scheme} which takes a secret key
        and encrypts messages based on a keyed-\gls{permutation}.}
}

\newglossaryentry{mac}
{
    name={message authentication code},
    description={A method in \gls{symmetriccrypto} for ensuring
        message integrity.
        MACs do not allow for nonrepudiation;
        for nonrepudiation, see \gls{signature}.},
}

\newglossaryentry{hmac}
{
    name={Hash-based Message Authentication Code},
    description={A general method for constructing
        a \gls{mac} from a \gls{hash function}.},
}

\newglossaryentry{ae}
{
    name={authenticated encryption},
    description={A method for combining an \gls{encryption scheme} with
        a \gls{mac}.},
}

\newglossaryentry{merkle tree}
{
    name={Merkle tree},
    description={A data structure consisting of leaves and nodes
        which allow for efficient proofs of inclusion.
        Uses a \gls{hash function} for security.},
}

\newglossaryentry{smt}
{
    name={Sparse Merkle tree},
    description={A \gls{merkle tree} where most of the leaves
        are empty.},
}

\newglossaryentry{smart contract}
{
    name={smart contract},
    description={A smart contract is a program which may automatically
        perform actions in response to changes in state.
        Smart contracts may be implemented on blockchains such as
        \gls{ethereum}.}
}

\newglossaryentry{ethereum}
{
    name={Ethereum},
    description={A blockchain with a built-in computational engine.
        It is possible to write \glspl{smart contract} within Ethereum.}
}

\newglossaryentry{ecc}
{
    name={Elliptic Curve Cryptography},
    description={The area of cryptography which uses \glspl{elliptic curve}
        as the basis for security.
        Security is based on the assumption that certain
        mathematical operations involving \glspl{elliptic curve}
        are impractical to solve.}
}

\newglossaryentry{symmetriccrypto}
{
    name={Symmetric Key Cryptography},
    description={The area of cryptography which uses one secret key.}
}

\newglossaryentry{publiccrypto}
{
    name={Public Key Cryptography},
    description={The area of cryptography which uses a two keys:
        a public key and a private key.
        The public key is derived from the private key,
        and it should be impractical to derive the private key
        from the public key.}
}

\newglossaryentry{pairingcrypto}
{
    name={Pairing-Based Cryptography},
    description={The area of cryptography which uses \glspl{bilinear}.}
}

\newglossaryentry{bilinear}
{
    name={bilinear pairing},
    description={A type of \gls{function} which is linear
        in both of its arguments;
        used extensively in \gls{pairingcrypto}.}
}

\newglossaryentry{zkproof}
{
    name={zero-knowledge proof},
    description={A proof in which no additional information
        is gained other than the fact that the statement is true.
        \Glspl{signature} are a standard example of zero-knowledge
        proofs.},
}

\newglossaryentry{recursion}
{
    name={recursion},
    description={See \gls{recursion}.}
}

%%%%%%%%%%%%%%%%%%%%%%%%%%%%%%%%%%%%%%%%%%%%%%%%%%%%%%%%%%%%%%%%%%%%%%%%
%%% Mathematics

\newglossaryentry{number theory}
{
    name={number theory},
    description={The division of mathematics devoted to studying
        the integers.
        This includes the study of prime numbers and related topics.}
}

\newglossaryentry{elliptic curve}
{
    name={elliptic curve},
    description={Defined over a \gls{field}, an elliptic curve is a \gls{group}
        defined by a collection of points which satisfy a certain polynomial
        equation as well as a group operation.
        Elliptic curves are used in \gls{ecc}.}
}

\newglossaryentry{group}
{
    name={group},
    description={A group is a \gls{set} together with a binary operation
        which satisfies additional properties.
        Standard examples of groups include the integers under addition
        as well as the positive rationals under multiplication.}
}

\newglossaryentry{subgroup}
{
    name={subgroup},
    description={A \gls{group} which is a subset of a larger group
        when using the inherited group operation.},
}

\newglossaryentry{cyclic group}
{
    name={cyclic group},
    description={A \gls{group} which is generated by one element;
        that is, $\parens{G,\cdot}$ is a cyclic group if
        $\parens{G,\cdot}$ is a \gls{group} and there is a $g\in G$
        such that for all $h\in G$ there is an $x\in\Z$ such that
        $h=g^{x}$.},
}

\newglossaryentry{abelian group}
{
    name={abelian group},
    description={A \gls{group} in which the underlying group operation
        is \gls{commutative}.},
}

\newglossaryentry{finite group}
{
    name={finite group},
    description={A \gls{group} which has a finite number of elements.},
}

\newglossaryentry{finite cyclic group}
{
    name={finite cyclic group},
    description={A \gls{group} which is both finite and cyclic.},
}

\newglossaryentry{ring}
{
    name={ring},
    description={A ring is a \gls{set} together with two binary operations
        (addition and multiplication) which satisfies additional properties.
        A standard example of rings include the integers under addition
        and multiplication.},
}

\newglossaryentry{commutative ring}
{
    name={commutative ring},
    description={A \gls{ring} in which multiplication is \gls{commutative}.},
}

\newglossaryentry{field}
{
    name={field},
    description={A field is a \gls{set} together with two binary operations
        (addition and multiplication) with additional properties;
        importantly, every nonzero element has a multiplicative inverse.
        A standard example of a field is the rational numbers
        under addition and multiplication.}
}

\newglossaryentry{finite field}
{
    name={finite field},
    description={A \gls{field} which has a finite number of elements.},
}

\newglossaryentry{commutative}
{
    name={commutative},
    description={The binary operation $\boxplus$ is commutative if
        $a\boxplus b = b\boxplus a$ for all $a$ and $b$.
        Standard examples of commutative operations include addition
        and multiplication;
        standard nonexamples include subtraction and division.
        Informally, an operation is commutative if
        ``the order of evaluation does not matter''.}
}

\newglossaryentry{associative}
{
    name={associative},
    description={The binary operation $\boxplus$ is associative if
        $\parens{a\boxplus b}\boxplus c = a\boxplus \parens{b\boxplus c}$
        for all $a$, $b$, and $c$.
        Standard examples include addition and multiplication;
        standard nonexamples include subtraction and division.
        Informally, an operation is associative if parentheses may be
        arbitrarily rearranged.}
}

\newglossaryentry{set}
{
    name={set},
    description={A set is a collection of objects.},
}

\newglossaryentry{function}
{
    name={function},
    description={A mapping which assigns an input from the domain
        to an output from the codomain.}
}

\newglossaryentry{injective}
{
    name={injective},
    description={A function $f$ is injective if $f(x) = f(y)$
        implies $x=y$; also called \emph{one-to-one}.}
}

\newglossaryentry{surjective}
{
    name={surjective},
    description={A function $f:A\to B$ is surjective if for $b\in B$
        there is an $a\in A$ such that $f(a)=b$; also called \emph{onto}.}
}

\newglossaryentry{bijective}
{
    name={bijective},
    description={A function $f$ is bijective (or is a bijection) if it is
        both \gls{injective} and \gls{surjective}.
        In this case, $f$ has an inverse function.}
}

\newglossaryentry{permutation}
{
    name={permutation},
    description={A function $f:A\to A$ is a permutation if $f$
        is \gls{bijective}.}
}

\newglossaryentry{discrete log}
{
    name={discrete logarithm},
    description={For a \gls{cyclic group} $G = \angles{g}$
        and $h\in G$, if $h = g^{x}$, then $x\in\Z$ is the discrete logarithm
        of $h$ with respect to $g$.},
}

\newglossaryentry{dlp}
{
    name={Discrete Logarithm Problem},
    description={Given a \gls{finite cyclic group} $G = \angles{g}$
        and $h\in G$, compute $x\in\Z$ such that $h = g^{x}$.},
}

\newglossaryentry{dhke}
{
    name={Diffie-Hellman Key Exchange},
    description={A key exchange in \gls{publiccrypto}
        based on deriving a \gls{shared secret} from public information.
        This comes from deriving a \gls{shared secret} within a
        \gls{cyclic group}.},
}

\newglossaryentry{dhp}
{
    name={Diffie-Hellman Problem},
    description={Given a \gls{finite cyclic group} $G = \angles{g}$
        of order $q$ with $a,b\chooseRandom{}\Z_{q}^{*}$,
        $A=g^{a}$, and $B=g^{b}$,
        solve for $g^{ab}$ given $A$ and $B$.},
}

\newglossaryentry{lagrange interpolation}
{
    name={Lagrange Interpolation},
    description={A method of interpolation using Lagrange polynomials.
        Interpolation methods are used throughout mathematics, science,
        and engineering.
        Given distinct nodes with associated values,
        the resulting polynomial is the polynomial of minimal degree
        which interpolates the result (agrees with the data).
        Within Cryptography, \emph{Lagrange Interpolation} is used
        within secret sharing protocols and \gls{distributed key generation}.}
}

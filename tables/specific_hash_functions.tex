\begin{table}[t]
\centering
\begin{tabular}{c|c|c|c|c}
\hline
\multicolumn{2}{c|}{\thead{Algorithm and \\ Variant}} &
\thead{Internal State \\ (bits)} &
\thead{Output Size \\ (bits)} &
\thead{Block Size \\ (bits)} \\
\hline
\multicolumn{2}{c|}{\MDFive{}} & 128 & 128 & 512 \\
\hline
\multicolumn{2}{c|}{\ShaOne{}} & 160 & 160 & 512 \\
\hline
\multirow{2}{*}{\ShaTwo{}} & \ShaTwo{}-256 & 256 & 256 & 512 \\
& \ShaTwo{}-512 & 512 & 512 & 1024 \\
\hline
\multirow{2}{*}{\ShaThree{}} & \ShaThree{}-256 & 1600 & 256 & 1088 \\
& \ShaThree{}-512 & 1600 & 512 & 576 \\
\hline
\end{tabular}
\caption[Comparison of Cryptographic Hash Functions]{Comparison
    of various \glsfirstplural{hash function}.
    We know that \MDFive{}, \ShaOne{}, and \ShaTwo{}
    are all based on the \MD{} construction;
    thus, their internal state and output size are the same.
    \ShaThree{} is based on sponge functions;
    its internal state is 1600 bits;
    the block size (rate) depends on the output.
    In the case of \ShaThree{}-256, the capacity is $512=2\cdot256$
    while the rate is $1088 = 1600-512$;
    for \ShaThree{}-512, the capacity is $1024=2\cdot512$
    while the rate is $576 = 1600-1024$.
    This table is based on the table from
    information from~\cite{rfc1321}, \cite[Figure 1]{FIPS-180-4-2015},
    and \cite[Section 6.1]{FIPS-202}.
}
\label{table:hash_functions}
\end{table}

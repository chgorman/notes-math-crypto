\begin{table}[t]
\centering
\begin{tabular}{c|c|c}
\hline
\multirowthead{3}{Effective \\ Key Length} &
    \multicolumn{2}{c}{\thead{Discrete Logarithm}} \\
\cline{2-3}
& \thead{Order-$q$ \\ Subgroup of $\F_{p}^{*}$} &
    \thead{Elliptic Curve \\ Group Order $q$} \\
\hline
112 & $p$: 2048,  $q$: 224 & 224 \\
128 & $p$: 3072,  $q$: 256 & 256 \\
192 & $p$: 7680,  $q$: 384 & 384 \\
256 & $p$: 15360, $q$: 512 & 512 \\
\hline
\end{tabular}
\caption[Key Length Estimates]{Estimated key lengths for specified
    security levels~\cite[Page 381]{IntroModernCrypto}.
    The effective key length determines the required number of operations.
    An effective key length of 128 bits means that
    it will take approximately $2^{128}$ operations to break the cryptosystem.
    As we can see, the order of \gls{elliptic curve} \glspl{group}
    are \emph{significantly smaller} than those required for
    \glspl{subgroup} of $\F_{p}$.}
\label{table:key_lengths}
\end{table}
